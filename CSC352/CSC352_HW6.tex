\documentclass{article}
\title{CSC352 HW6}
\author{Alex Zhang}
\date{Feb 2023}

\textwidth=16.00cm 
\textheight=22.00cm 
\topmargin=0.00cm
\oddsidemargin=0.00cm 
\evensidemargin=0.00cm 
\headheight=0cm 
\headsep=0.5cm
\textheight=610pt

\usepackage{latexsym,array,delarray,amsthm,amssymb,epsfig}
\usepackage{amsmath}

\newcommand{\bmat}[1]{\begin{bmatrix} #1 \end{bmatrix}}
\newcommand{\mat}[1]{\mathbf{#1}}

\let\ds\displaystyle

\begin{document}
\maketitle

\section*{Question 1}
Given that $\tilde{x} = x(1+\varepsilon_x)$, and $\tilde{y} = y(1+\varepsilon_y)$, we can simplify the inequality,
\begin{align}
    \left| \frac{xy-\tilde{x}\tilde{y}}{xy}\right| &\leq (2+\varepsilon)\varepsilon \nonumber \\
    \left| \frac{xy-(xy+xy\varepsilon_x+xy\varepsilon_y+xy\varepsilon_x\varepsilon_y)}{xy}\right| &\leq (2+\varepsilon)\varepsilon \nonumber \\
    \left| \frac{-xy\varepsilon_x-xy\varepsilon_y-xy\varepsilon_x\varepsilon_y}{xy}\right| &\leq (2+\varepsilon)\varepsilon \nonumber \\
    \left| -\varepsilon_x-\varepsilon_y-\varepsilon_x\varepsilon_y\right| &\leq (2+\varepsilon)\varepsilon \nonumber \\
    \left| \varepsilon_x+\varepsilon_y+\varepsilon_x\varepsilon_y\right| &\leq \varepsilon + \varepsilon + \varepsilon^2 \nonumber
\end{align}
\paragraph*{Csae 1:} $\varepsilon = \left|\frac{x-\tilde{x}}{x}\right|$\\
Since $\tilde{x} = x(1+\varepsilon_x)$, $\varepsilon_x = \left|\frac{\tilde{x}-x}{x}\right| =  \left|\frac{x-\tilde{x}}{x}\right| $. Indicate
$\varepsilon = \varepsilon_x$. Because $\varepsilon_x \geq \varepsilon_y$, $\varepsilon_x^2 \geq \varepsilon_x \varepsilon_y$, we can get that,
$$\varepsilon_x + \varepsilon_x + \varepsilon_x^2 \geq \left| \varepsilon_x+\varepsilon_y+\varepsilon_x\varepsilon_y\right| $$
which is the same as the simplified inequality.

\paragraph*{Case 2:} $\varepsilon = \left|\frac{y-\tilde{y}}{y}\right|$\\
Without loss of generality, we can apply the same proof on $\varepsilon_y$ using $\varepsilon_x$'s and it will have the same result. $\blacksquare$




\section*{Question 3}

\subsection*{(a)}
The solution for $\mat{x}$ is $\bmat{-4199\\8601\\-4699}$.

\subsection*{(b)}
The solution for $\mat{x}$ is $\bmat{-4438\\9090\\-4966}$.


\subsection*{(c)}
I think it is ill-conditioned because I changed one entry by subtracting
one, but my results vary from an absolute value about 400.


\subsection*{(d)}
The value of condition number is about $65886$, which is large. I think my assumption about
the ill-conditioned holds true because the condition number 




\section*{Question 4}

\subsection*{(a)}
The first component is fraction $f$, the second one is exponent $e$, and the thrid one 
is the sign of this number. In double precision, every number can be represented as,
$$\pm (1+f) \cdot 2^{e}$$


\subsection*{(b)}

Following is the floating point represeation of decimal number $-12$,

$$(-1)^1(1+0.5)\cdot 2^{3}$$


\subsection*{(c)}
The biggest possible floating point number shoule be,
$$(-1)^0(1+(1-2^{-52})) \cdot 2^{1023} = (2-2^{-52})\cdot 2^{1023}$$
The smallest possible positive floating point number will be,
$$(-1)^0 (1)\cdot 2^{-1022} = 2^{-1022}$$


\subsection*{(d)}
By definition, machine epsilon is the distance from 1 to the next larger floating point number.
In term of floating point representation,
$$eps = |(1+\min f) \cdot 2^0 - 1 |= \min f$$
The value will be $\min  f = 2^(-52) \approx 2.22e-16$










\end{document}