\documentclass{article}
\title{CSC352 HW7}
\author{Alex Zhang}
\date{March 2023}
\textwidth=16.00cm 
\textheight=22.00cm 
\topmargin=0.00cm
\oddsidemargin=0.00cm 
\evensidemargin=0.00cm 
\headheight=0cm 
\headsep=0.5cm
\textheight=610pt

\usepackage{latexsym,array,delarray,amsthm,amssymb,epsfig}
\usepackage{amsmath}

\newcommand{\bmat}[1]{\begin{bmatrix} #1 \end{bmatrix}}
\newcommand{\mat}[1]{\mathbf{#1}}

\let\ds\displaystyle

\begin{document}
\maketitle

\section*{Question 1}
    \begin{enumerate}
        \item The relative error for $\mat{Q}$ is $2$.
        \item The relative error for $\mat{R}$ is $1.036$.
        \item The relative error for $\mat{Q*R}$ is $1.1194e-15$.
    \end{enumerate}
    I'm surprised with the first two relative errors. For $\mat{Q}$'s relative error, it should be $0$ ideally,
    but I got $2$, which shows there is a difference between true $\mat{Q}$, and calculated $\mat{Q}$. For $\mat{R}$, 
    I think it still should be $0$ for $\| 0\|_p = 0$. Base on the two relative errors, I think for HouseHolder QR, $\mat{Q}$ and $\mat{R}$ are not accurate.
    However, the relative error for $\mat{Q*R}$ is really small so their product is accurate. Based on this
    small relative error, we can also conclude that QR factorization using HouseHolder is stable.

\section*{Question 2}
    \begin{enumerate}
        \item For QR factorization with HouseHolder, the distance is $8.6905e-16$.
        \item For QR factorization with modified Gram-Schmidt, the distance is $1$.
    \end{enumerate}
    The distance using HouseHolder is very small and therefore reasonable. However, the result for using mgs is quiet big.
    One reason is when doing Gram-Schmidt process, calculating matrix $\mat{Q}$ involves multiplication and normalization.
    This process will make $\mat{Q}$ not be strictly orthonormal matrix, and the result will be affected then. 
\end{document}