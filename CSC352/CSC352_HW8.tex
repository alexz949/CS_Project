\documentclass{article}
\title{CSC352 HW8}
\author{Alex Zhang}
\date{April 2023}
\textwidth=16.00cm 
\textheight=22.00cm 
\topmargin=0.00cm
\oddsidemargin=0.00cm 
\evensidemargin=0.00cm 
\headheight=0cm 
\headsep=0.5cm
\textheight=610pt
\usepackage{graphicx}

\graphicspath{ {./images/} }

\usepackage{latexsym,array,delarray,amsthm,amssymb,epsfig,amsmath,listings}

\lstset{
  basicstyle=\ttfamily,
  mathescape
}

\newcommand{\bmat}[1]{\begin{bmatrix} #1 \end{bmatrix}}
\newcommand{\mat}[1]{\mathbf{#1}}

\let\ds\displaystyle

\begin{document}
\maketitle
\section*{Question 2}
I created a file "my\textunderscore chol.m" using recursive function.

\section*{Question 3}
Since $\mat{A} = \bmat{2 & -1 & 0\\
                        -1 & 2 & -1\\
                        0 & -1 & 2}$
, then $\mat{A - \lambda \cdot I} =\bmat{2-\lambda & -1 & 0\\
-1 & 2-\lambda & -1\\
0 & -1 & 2-\lambda}$ and the corresponding characteristis polynomial will be,
\begin{align}
    \det(\mat{A - \lambda \cdot I}) &= (2-\lambda)(2-\lambda)(2-\lambda) -(2-\lambda)(-1)(-1) - (-1)(-1)(2-\lambda) + 0  + 0 - 0 \nonumber \\ 
    &= (2-\lambda)^3 - (2-\lambda) - (2-\lambda) \nonumber \\
    &= (2-\lambda)^3 - 2 \cdot (2-\lambda) \nonumber \\
    &= -\lambda^3 + 6\lambda^2 - 10 \lambda + 4 \nonumber
\end{align}
Let it be 0,
\begin{align}
    -\lambda^3 + 6\lambda^2 - 10 \lambda + 4 &= 0 \nonumber \\
    -(\lambda-2)(\lambda^2 - 4\lambda +2) &= 0 \nonumber
\end{align}
We can get that $\lambda_1 = 2+ \sqrt{2}$, $\lambda_2 = 2$, and $\lambda_3 = 2- \sqrt{2}$. Given that all eigenvalues
are positive and $\mat{A}$ is symmetric, $\mat{A}$ is a symmetric positive definite matrix.





\section*{Question 4}
The pseudocode will be:
\begin{lstlisting}
    function x = tri_mat(M,y)
        n = size(M,1)
        x = zeros(n,1)
        for i = 2 to n
            $l = a_i / b_{i-1}$
            $b_i = b_i - l \cdot c_{i-1}$
            $y_i = y_i - l \cdot y_{i-1}$
        end for
        $x_n = y_n / b_n$
        for j = n-1 to 1
            $x_j = (y_j-c_{j}\cdot x_{j+1})/b_j$
        end for
    end fucntion
\end{lstlisting}
The algorithm will first perform gaussian elimination, thus creating a new upper triangular matrix.
After that uses back substituion to solve the linear system. In first loop, each iteration does 5 flops and 
there are total $n-1$ iterations, so the total flops will be $5n - 5$.
The second loop contains 3 flops in each iteration and the total flops will be $3n -3$. The time complexity 
for this algorithm will be $O(8n-8) = O(n)$.
\end{document}