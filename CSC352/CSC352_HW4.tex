\documentclass{article}
\title{CSC352 HW4}
\author{Alex Zhang}
\date{Feb 2023}


\textwidth=16.00cm 
\textheight=22.00cm 
\topmargin=0.00cm
\oddsidemargin=0.00cm 
\evensidemargin=0.00cm 
\headheight=0cm 
\headsep=0.5cm
\textheight=610pt

\usepackage{latexsym,array,delarray,amsthm,amssymb,epsfig}
\usepackage{amsmath}

\newcommand{\bmat}[1]{\begin{bmatrix} #1 \end{bmatrix}}
\newcommand{\mat}[1]{\mathbf{#1}}

\let\ds\displaystyle

\begin{document}
\maketitle

\section*{Question 1}
Since $\mat{A} = \bmat{1 & 1 & 3 \\ 0 & 2 & 1 \\ 0 & 0 & 1 \\ -1 & -1 & -1}$,
let $\mat{q_1} = \frac{\mat{a_1}}{\| \mat{a_1}\|_2}  = \bmat{1/\sqrt{2} \\ 0 \\ 0 \\ -1/ \sqrt{2}}$.
In this case,
\begin{align}
    \mat{q_2} &= \mat{a_2}-(\mat{q_1}^\top\mat{a_2})\mat{q_1} \nonumber \\
    \mat{q_2} &= \bmat{1\\2\\0\\-1}-(1/\sqrt{2} + 1/\sqrt{2}) \bmat{1/\sqrt{2} \\ 0 \\ 0 \\ -1/ \sqrt{2}} \nonumber \\
    \mat{q_2} &= \bmat{0\\2\\0\\0} \nonumber
\end{align}
Normalizing $\mat{q_2}$,
$$\mat{q_2} = \bmat{0\\1\\0\\0}$$
For $\mat{q_3}$,
\begin{align}
    \mat{q_3} &= \mat{a_3} - (\mat{q_1}^\top \mat{a_3})\mat{q_1} - (\mat{q_2}^\top \mat{a_3})\mat{q_2} \nonumber \\
    \mat{q_3} &= \bmat{3\\1\\1\\-1} - (3/\sqrt{2} + 1\sqrt{2})\bmat{1/\sqrt{2}\\0\\0\\-1/\sqrt{2}} - (\mat{q_2}^\top \mat{a_3})\mat{q_2} \nonumber \\
    \mat{q_3} &= \bmat{3\\1\\1\\-1} - (3/\sqrt{2} + 1\sqrt{2})\bmat{1/\sqrt{2}\\0\\0\\-1/\sqrt{2}} - (1)\bmat{0\\1\\0\\0} \nonumber \\
    \mat{q_3} &= \bmat{1\\0\\1\\1} \nonumber
\end{align}
Normalizing $\mat{q_3}$,
$$\mat{q_3} = \bmat{1/ \sqrt{3} \\ 0 \\ 1/\sqrt{3} \\ 1/\sqrt{3}}$$
So $\mat{Q} = \bmat{1/\sqrt{2} & 0 & 1/\sqrt{3} \\ 0 & 1 & 0 \\ 0 & 0 & 1/\sqrt{3} \\ -1/ \sqrt{2} & 0 & 1/\sqrt{3}}$, 
And for $\mat{R}$, $\mat{r_{ij} = \mat{q_i}^\top}\mat{a_j}$, $\mat{r_{jj}} = \| \mat{a_j} - \sum^{j-1}_{i=1}\mat{r_{ij}\mat{q_i}} \|_2$, 
$$\mat{R} = \begin{bmatrix}
    \sqrt{2} & \sqrt{2} & 2\sqrt{2} \\
    0 & 2 & 1 \\
    0 & 0 & \sqrt{3}
\end{bmatrix}$$
The reduced $\mat{QR}$ decomposition will be $$A = \bmat{1/\sqrt{2} & 0 & 1/\sqrt{3} \\ 0 & 1 & 0 \\ 0 & 0 & 1/\sqrt{3} \\ -1/ \sqrt{2} & 0 & 1/\sqrt{3}}  \begin{bmatrix}
    \sqrt{2} & \sqrt{2} & 2\sqrt{2} \\
    0 & 2 & 1 \\
    0 & 0 & \sqrt{3}
\end{bmatrix}  $$

\section*{Question 2}
    \subsection*{(a)}
    For a matrix  $\mat{X}$, and for a vector $\mat{v}$, their multiplication will be,
    $$\mat{X}\mat{v} = \begin{bmatrix}
        x_{11} & x_{12} & \dots & x_{1n} \\
        x_{21} & x_{22} & \dots & x_{2n} \\
        \vdots & \vdots & \ddots & \vdots \\
        x_{m1} & x_{m2} & \dots & x_{mn} 
    \end{bmatrix} \bmat{v_1 \\ v_2 \\ \vdots \\ v_n} = \begin{bmatrix}
        x_{11}v_1 + x_{12}v_2 + \dots  + x_{1n}v_n\\
        x_{21}v_1 + x_{22}v_2 + \dots  + x_{2n}v_n\\
        \vdots \\
        x_{m1}v_1 + x_{m2}v_2 + \dots  + x_{mn}v_n

    \end{bmatrix}$$
    There are $n$ times of multiplication and $n-1$ addition in each row. The total number of flops
    in each row is $(2n-1)$. Since there are total $m$ rows, the total number of flops will be $$2mn - m$$
    
    \subsection*{(b)}
    For a matrix $\mat{X}$ and a matrix $\mat{Y}$, assume their multiplication will be matrix $\mat{A}$.
    $$\mat{X}\mat{Y} =  \begin{bmatrix}
        x_{11} & x_{12} & \dots & x_{1n} \\
        x_{21} & x_{22} & \dots & x_{2n} \\
        \vdots & \vdots & \ddots & \vdots \\
        x_{m1} & x_{m2} & \dots & x_{mn} 
    \end{bmatrix} \begin{bmatrix}
        y_{11} & y_{12} & \dots & y_{1p} \\
        y_{21} & y_{22} & \dots & y_{2p} \\
        \vdots & \vdots & \ddots & \vdots \\
        y_{n1} & y_{n2} & \dots & y_{np} 
        
    \end{bmatrix} =A $$
    Because of the matrix multiplication rule, 
    $$\mat{a}_{ij} = x_{i1}y_{1j} + x_{21}y_{2j} + \dots + x_{in}y_{nj}$$
    And since $\mat{A} \in \mathbb{R}^{m \times p}$, for matrix $\mat{A}$, there are total $mp$ entries.
    for each entries, the number of flops will be $n$ multiplication and $n-1$ addition, which is $2n-1$.\\
    So the total number of flops for matrix times matrix will be $$2nmp - mp$$
    \subsection*{(c)}
    Given a matrix $\mat{X} \in \mathbb{R}^{m \times n}$, the product of its transpose and itself $\mat{X}^\top \mat{X}$
    be a matrix $\mat{C}$, and each entry of $\mat{C}$ also follows that
     $$\mat{c}_{ij} = x_{i1}x_{1j} + x_{21}x_{2j} + \dots + x_{im}x_{mj}$$
    However, $\mat{C}$ is a symmetric matrix since it equals $\mat{X}^\top \mat{X}$, which means we only need to calculate
    the upper right side and the main diagonal. There are total $(mn/2 + m/2)$ entries. For each entries, the number of
    flops will be $2m-1$, so the total number of flops will be 
    $$m^2n + m^2 -mn/2 -m/2$$
    \section*{Question 3}
    

\end{document}