\documentclass{article}
\usepackage[T1]{fontenc}
\usepackage[UTF8]{ctex}
\title{CSC352 HW5}
\author{Alex Zhang}
\date{Feb 2023}



\textwidth=16.00cm 
\textheight=22.00cm 
\topmargin=0.00cm
\oddsidemargin=0.00cm 
\evensidemargin=0.00cm 
\headheight=0cm 
\headsep=0.5cm
\textheight=610pt

\usepackage{latexsym,array,delarray,amsthm,amssymb,epsfig}
\usepackage{amsmath}

\newcommand{\bmat}[1]{\begin{bmatrix} #1 \end{bmatrix}}
\newcommand{\mat}[1]{\mathbf{#1}}

\let\ds\displaystyle



\begin{document}
\maketitle

\section*{Question 1}

\subsection*{(a)}
Since $\mat{x} \in \mathbb{R}^m$, let $\mat{q_1} = \frac{\mat{x_1}}{\|\mat{x_1}\|_2}$. Because $\mat{x}$ is a vector, it only has one column, the
matrix $\mat{Q}$ is just $\mat{q_1}$. For $\mat{R}$, since there is only one columne, $\mat{R} = \mat{r}_{11} = \|\mat{x_1} \|_2$. The QR
decomposition will be 
$$\mat{x} = \frac{\mat{x_1}}{\|\mat{x_1}\|_2} \cdot  \|\mat{x_1} \|_2$$

\subsection*{(b)}
Given and orthogonal matrix $\mat{J} \in \mathbb{R}^{m \times n}$. Let $\mat{q}_1 = \frac{\mat{j}_1}{\|\mat{j}_1\|_2}$, and $\mat{q}_2$ is,
$$\mat{q}_2= \frac{\mat{j}_2-(\mat{q}_1^\top\mat{j}_2)\mat{q}_1}{\|\mat{j}_2-(\mat{q}_1^\top\mat{j}_2)\mat{q}_1\|_2}$$
Since $\mat{J}$ is an orthogonal matrix, $\mat{q}_1 \cdot \mat{j}_2 = 0$, indicating
$$\mat{q}_2 = \frac{\mat{j}_2}{\|\mat{j}_2\|_2}$$
This case can be generalized into any column $\mat{q}_i$ for matrix $\mat{Q}$. For matrix $\mat{R}$, $\mat{r}_{ij} = \mat{q}_i^\top \mat{j}_j$.
In this case, since $\mat{Q}$ is orthonormal matrix to $\mat{J}$, $\mat{r}_{ij} = 0$ $ \forall i \mbox{, } j \leq m \mbox{, } n$. And for diagonal entries
$$\mat{r}_{jj} = \|\mat{j}_j -\sum^{i-1}_{i=1}\mat{r}_{ij}\mat{q}_i\|_2$$
Where $\sum^{i-1}_{i=1}\mat{r}_{ij}\mat{q}_i\|_2 = 0$ because $\mat{r}_{ij} = 0$. Therefore, $\mat{r}_{jj} = \|\mat{j}_j\|_2$.

Overall, after doing QR decomposition on an orthogonal matrix, we get $\mat{Q}$ is an orthonormal matrix and each column is the normal vector from $\mat{J}$.
$\mat{R}$ is a diagonal matrix with each entry represents the 2-norm of the corresponding column vector in $\mat{J}$.








orthgonal. If it is orthgonal, Q is normalizing each (general proof needed). and the R is having the norm on main diagonal.


\subsection*{(c)}
upper traingular. This will make Q to be a identity matrix and R is just the original matrix (proof needed).

\subsection*{Question 2}









\end{document}