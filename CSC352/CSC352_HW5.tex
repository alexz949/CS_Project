\documentclass{article}
\usepackage[T1]{fontenc}
\usepackage[UTF8]{ctex}
\title{CSC352 HW5}
\author{Alex Zhang}
\date{Feb 2023}



\textwidth=16.00cm 
\textheight=22.00cm 
\topmargin=0.00cm
\oddsidemargin=0.00cm 
\evensidemargin=0.00cm 
\headheight=0cm 
\headsep=0.5cm
\textheight=610pt

\usepackage{latexsym,array,delarray,amsthm,amssymb,epsfig}
\usepackage{amsmath}

\newcommand{\bmat}[1]{\begin{bmatrix} #1 \end{bmatrix}}
\newcommand{\mat}[1]{\mathbf{#1}}

\let\ds\displaystyle



\begin{document}
\maketitle

\section*{Question 1}

\subsection*{(a)}
Since $\mat{x} \in \mathbb{R}^m$, let $\mat{q_1} = \frac{\mat{x_1}}{\|\mat{x_1}\|_2}$. Because $\mat{x}$ is a vector, it only has one column, the
matrix $\mat{Q}$ is just $\mat{q_1}$. For $\mat{R}$, since there is only one columne, $\mat{R} = \mat{r}_{11} = \|\mat{x_1} \|_2$. The QR
decomposition will be 
$$\mat{x} = \mat{Q}\mat{R} = \frac{\mat{x_1}}{\|\mat{x_1}\|_2} \cdot  \|\mat{x_1} \|_2$$

\subsection*{(b)}
Given and orthogonal matrix $\mat{J} \in \mathbb{R}^{m \times n}$. Let $\mat{q}_1 = \frac{\mat{j}_1}{\|\mat{j}_1\|_2}$, and $\mat{q}_2$ is,
$$\mat{q}_2= \frac{\mat{j}_2-(\mat{q}_1^\top\mat{j}_2)\mat{q}_1}{\|\mat{j}_2-(\mat{q}_1^\top\mat{j}_2)\mat{q}_1\|_2}$$
Since $\mat{J}$ is an orthogonal matrix, $\mat{q}_1 \cdot \mat{j}_2 = 0$, indicating
$$\mat{q}_2 = \frac{\mat{j}_2}{\|\mat{j}_2\|_2}$$
This case can be generalized into any column $\mat{q}_i$ for matrix $\mat{Q}$. For matrix $\mat{R}$, $\mat{r}_{ij} = \mat{q}_i^\top \mat{j}_j$.
In this case, since $\mat{Q}$ is orthonormal matrix to $\mat{J}$, $\mat{r}_{ij} = 0$ $ \forall i \mbox{, } j \leq m \mbox{, } n$. And for diagonal entries
$$\mat{r}_{jj} = \|\mat{j}_j -\sum^{i-1}_{i=1}\mat{r}_{ij}\mat{q}_i\|_2$$
Where $\sum^{i-1}_{i=1}\mat{r}_{ij}\mat{q}_i\|_2 = 0$ because $\mat{r}_{ij} = 0$. Therefore, $\mat{r}_{jj} = \|\mat{j}_j\|_2$.

Overall, after doing QR decomposition on an orthogonal matrix, we get $\mat{Q}$ is an orthonormal matrix and each column is the normal vector from $\mat{J}$.
$\mat{R}$ is a diagonal matrix with each entry represents the 2-norm of the corresponding column vector in $\mat{J}$.







\subsection*{(c)}
Given an upper triangular matrix $\mat{T} \in \mathbb{R}^{m \times n}$
Let $\mat{q_1} = \frac{\mat{t_1}}{\|\mat{t_1}\|_2}$, which is just $\mat{e_1}$ in this case. For $\mat{q_2}$, it should be
\begin{align}
    \mat{q}_2 &= \frac{\mat{t}_2-(\mat{q}_1^\top\mat{t}_2)\mat{q}_1}{\|\mat{t}_2-(\mat{q}_1^\top\mat{t}_2)\mat{q}_1\|_2} \nonumber \\
    \mat{q}_2 &= \frac{\bmat{0 \\ \mat{t_22} \\ \vdots \\ 0}}{\|\bmat{0 \\ \mat{t_22} \\ \vdots \\ 0}\|_2} \nonumber 
\end{align}
which is just $\mat{e_2}$. We can then general all column $\mat{q_i}$ in matrix $\mat{Q}$ as $\mat{e_i}$. Therefore, $\mat{Q}$ is just an
identity matrix with dimention $m \times m$.

Based on the informaiton that $\mat{Q}$ is an identity matrix, we can then just get $\mat{R}$ which $\mat{R}$ is just 
the original matrix since a matrix times an identity matrix will still be itself.
$$\mat{T} = \mat{Q}\mat{R} = \mat{I}\mat{T}$$



\subsection*{Question 2}
Let $\mat{u_1} = \mat{a_1}+\|\mat{a_1}\|_2\mat{e_1} = \bmat{4\\2\\-2\\1} + 5\bmat{1\\0\\0\\0} = \bmat{9\\2\\-2\\1}$.\\
new $\mat{a_1} = \bmat{4\\2\\-2\\1} -2 \dfrac{45}{90} \bmat{9\\2\\-2\\1} = \bmat{-5\\0\\0\\0} $\\
new $\mat{a_2} = \bmat{-3\\-14\\14\\-7}- 2 \dfrac{-90}{90}\bmat{9\\2\\-2\\1} =\bmat{15\\-10\\10\\-5}$, which is
$\bmat{-10\\10\\-5}$\\
new  $\mat{a_3} = \bmat{4\\-3\\0\\15}- 2 \dfrac{45}{90}\bmat{9\\2\\-2\\1} =\bmat{-5\\-5\\2\\14}$, which is $\bmat{-5\\2\\14}$.

Similar, $\mat{u_2} = \mat{a_2} - \|\mat{a_2}\|_2\mat{e_1} = \bmat{-10\\10\\-5} - 15\bmat{1\\0\\0} = \bmat{-25\\10\\-5}$\\
new $\mat{a_2} = \bmat{-10\\10\\-5} - 2\dfrac{375}{750}\cdot \bmat{-25\\10\\-5} = \bmat{15\\0\\0}$.\\
new $\mat{a_3} = \bmat{-5\\2\\14} - 2\dfrac{75}{750}\cdot \bmat{-25\\10\\-5} = \bmat{0\\0\\15}$, which is $\bmat{0\\15}$.

$\mat{u_3} = \mat{a_3} + \|\mat{a_3}\|_2\mat{e_1} =\bmat{0\\15} + 15\bmat{1\\0} = \bmat{15\\15}$\\
new $\mat{a_3} = \bmat{0\\15} - 2\dfrac{225}{450}\cdot \bmat{15\\15} = \bmat{-15\\0}$, therefore, we can get $\mat{R}$
$$\begin{bmatrix}
    -5 & 15 & -5 \\
    0 & 15 & 0 \\
    0 & 0 & -15\\
    0 & 0 & 0
\end{bmatrix}$$


\subsection*{Question 3}
Given $\mat{A}^\dagger = (\mat{A}^\top \mat{A})^{-1}\mat{A}^\top$,

\subsection*{(a)}
The left side can be transformed into:
\begin{align}
    \mat{A}\mat{A}^\dagger\mat{A} &= \mat{A}  (\mat{A}^\top \mat{A})^{-1}\mat{A}^\top \mat{A}\nonumber \\
    &= \mat{A}\mat{A}^{-1}(\mat{A}^\top)^{-1}\mat{A}^\top \mat{A} \nonumber
\end{align}
Since $\mat{A}\mat{A}^{-1} = \mat{I}$ and $(\mat{A}^\top)^{-1}\mat{A}^\top = \mat{I}$,
$$\mat{A}\mat{A}^\dagger\mat{A} = \mat{I}\mat{I}\mat{A} = \mat{A}$$
$\blacksquare$

\subsection*{(b)}











\end{document}