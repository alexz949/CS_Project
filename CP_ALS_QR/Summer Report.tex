\documentclass{article}
\title{CP-ALS-QR report}
\author{Alex Zhang}
\date{July 2023}
\textwidth=16.00cm 
\textheight=22.00cm 
\topmargin=0.00cm
\oddsidemargin=0.00cm 
\evensidemargin=0.00cm 
\headheight=0cm 
\headsep=0.5cm
\textheight=610pt
\usepackage{graphicx}
\usepackage{multicol}

\graphicspath{ {./images/} }

\usepackage{latexsym,array,delarray,amsthm,amssymb,epsfig}
\usepackage{amsmath}
\usepackage{listings}
\lstset{
  basicstyle=\ttfamily,
  mathescape
}

\newcommand{\bmat}[1]{\begin{bmatrix} #1 \end{bmatrix}}
\newcommand{\mat}[1]{\mathbf{#1}}

\let\ds\displaystyle

\begin{document}
\maketitle
\section{Introduction}
The CANDECOMP/PARAFAC or canonical polyadic (CP) decomposition for multidimensional data, 
or tensors, is a popular tool for analyzing and interpreting latent patterns that may be 
present in multidimensional data. Basically CP decomposition of a tensor refers to its
expression as a sum of $r$ rank-one components and each of them is a vector outer product.
One of the most popular methods used to compute a CP decomposition is the alternating least
squares (CP-ALS) approach, which solves a series of linear least squares problems. Usually
to solve these linear leaste squares problems, normal equations are used for CP-ALS. This
approach may be sensitive for ill-conditioned inputs. Based on this idea, there are already
a more stable approach which is solving the linear least sqaures problems using QR decomposition
instead.
\\
For my summer research project, I basically follows the QR apprach but assuming the input tensor
is in Kruskal structure, that is, a tensor stored as factor matrices and corresponding weights.
By exploiting this structure, we improve the computation efficiency by not forming Multi-TTM tensor.
The problem left is when doing CP-ALS, QR-based methods is exponential in $N$, the number of modes.







\end{document}