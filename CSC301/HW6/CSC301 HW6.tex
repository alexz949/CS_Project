\documentclass{article}
\title{CSC301 HW6}
\author{Alex Zhang}
\date{Feb 2023}
\textwidth=16.00cm 
\textheight=22.00cm 
\topmargin=0.00cm
\oddsidemargin=0.00cm 
\evensidemargin=0.00cm 
\headheight=0cm 
\headsep=0.5cm
\textheight=610pt

\usepackage{amsthm,amssymb}
\usepackage{amsmath}

\newcommand{\bmat}[1]{\begin{bmatrix} #1 \end{bmatrix}}
\newcommand{\mat}[1]{\mathbf{#1}}
\newcommand{\mb}[1]{\mathbb{#1}}

\let\ds\displaystyle



\begin{document}
\maketitle

\section*{Question 1}
For the algorithm, readFile method enables transforming text file into matrix represeations.
The following code blocks is creating the number of two-paths in adjacency list represeation.
In the last part of the main function, the BFS is used to find recommander and writing the file out.
I also have some comments on codes.



\section*{Question 2}
For the complexity analysis, my code complexity contains different parts.
The first part is reading the matrix I think the time complexity will be $O(m)$. The second part
is recording the number of two-paths in adjacency list representation which the time complexity will be $O(nd^2 + nd)$.
The third part is fidning the recommandation integer which the time complexity will be $O(nd\log d + n^2 + nm + 2nd)$. Overall 
the time complexity for this algorithm will be
$$O(n^2 + nd^2 + nd\log d + 3nd + nm + m)$$
Which I think in this case the upper bound should be $O(n^2)$ or $O(nd^2)$ depending on the number of $d$.

\section*{Question 3}


\subsection*{(a)}
When doing the A16.txt file, both adjacency list and adjacency matrix method have roughly same time. However,
when doing A1024.txt file, for adjacency list method, the time doing is about $231$ millisecond. The adjacency matrix method costs $94165$ millisecond.
This is a big differece. The adjacency list is faster.
I think the reason is he adjacency matrix method includes a matrix multiplication method when recommending, which the time complexity is $O(n^{2.81})$.
However, in adjacency list method based on my complexity analysis in question 2, it only costs $O(n^2)$ or $O(nd^2)$ which is way cheaper then 
matrix multiplication.

\subsection*{(b)}
Based on rec1024.txt file, I find one recommendation which the first word "existence" is recommended to the fourth word "unsubstantiality".
The intermediate word is the second word which is "inexistence". I think it is meaningful because we cannot say something exists if there is lack
of specific evidence. Or it is not substantial which we can hardly say existence.









\end{document}