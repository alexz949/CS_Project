\documentclass{article}
\title{CSC301 HW4}
\author{Alex Zhang}
\date{Feb 2023}



\textwidth=16.00cm 
\textheight=22.00cm 
\topmargin=0.00cm
\oddsidemargin=0.00cm 
\evensidemargin=0.00cm 
\headheight=0cm 
\headsep=0.5cm
\textheight=610pt

\usepackage{amsthm,amssymb}

\newcommand{\bmat}[1]{\begin{bmatrix} #1 \end{bmatrix}}
\newcommand{\mat}[1]{\mathbf{#1}}
\newcommand{\mb}[1]{\mathbb{#1}}

\let\ds\displaystyle


\begin{document}
\maketitle
\section*{Question 1}
\paragraph{Proof:} $\mat{B} = \mat{A}\mat{A} = \mat{A}^2$

If there exists a path of length two between vertex $u$ and $w$, then $\exists $ $i$ which 
in adjacency matrix $\mat{A}$, both $\mat{a}_{ui}$ and $\mat{a}_{iw}$ exists.
In order to ensure both $\mat{a}_{ui}$ and $\mat{a}_{iw}$ exists, their product $\mat{a}_{ui} \cdot \mat{a}_{iw}$
has to be 1.

Since $\mat{B}$ is counting the number of two-paths in the given graph. With given vertex $u$, and vertex $w$, $\mat{B}_{uw}$
is just the sum of all exist two-paths. In mathematical expression $$\mat{B}_{uw} = \sum^{n}_{i=0}\mat{a}_{ui}\cdot \mat{a}_{iw}$$
Also by the definition of matrix multiplication, 
$$\mat{A}\mat{A}_{uw} = \sum^{n}_{i=0}\mat{a}_{ui}\cdot \mat{a}_{iw}$$
for all vertex $u$ and $w$. This implies that,
$\mat{B} = \mat{A}\mat{A} = \mat{A}^2$.$\blacksquare$

\section*{Question 2}


\section*{Question 3}
\subsection*{(a)}
\begin{enumerate}
    \item For a, the best recommand should be  
\end{enumerate}








\end{document}