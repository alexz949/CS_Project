\documentclass{article}
\title{CSC301 HW8}
\author{Alex Zhang}
\date{March 2023}
\textwidth=16.00cm 
\textheight=22.00cm 
\topmargin=0.00cm
\oddsidemargin=0.00cm 
\evensidemargin=0.00cm 
\headheight=0cm 
\headsep=0.5cm
\textheight=610pt
\usepackage{graphicx}

\graphicspath{ {./images/} }

\usepackage{latexsym,array,delarray,amsthm,amssymb,epsfig}
\usepackage{amsmath}
\usepackage{listings}
\lstset{
  basicstyle=\ttfamily,
  mathescape
}

\newcommand{\bmat}[1]{\begin{bmatrix} #1 \end{bmatrix}}
\newcommand{\mat}[1]{\mathbf{#1}}

\let\ds\displaystyle

\begin{document}
\maketitle
\section*{Question 1}
The updated prgram is called "knapsack res" and there are three output files called 
"smallout.txt", "mediumout.txt", and "largeout.txt" which write the output in the format.

\section*{Question 2}
\paragraph{Space}

In the file "knapsackCap"'s knapscakCap method, I created two constant dimension arrays "bag"
and "m". Both of them are actaully $(W+1) \times 2$ matrices. So the overall space complexity will
be $O(2*2*(W+1)) = O(4W + 4)$. Since $W$ is the leading term and $4$ is a constant, the overall
space complexity will be
$$O(W)$$

\paragraph{Time}

In "knapsackCap" method, first there is one loop for initialization, which takes $O(W)$. Further,
there is a nested loop which the outer loop iterates $n$ times, and inner loop iterates $W$ times.
The time complexity for this loop is $O(nW)$. Assume that assigning variables and comparision take constant
time. The overall time complexity will be $O(W + nW)$, which is same as 
$$O(nW)$$




\section*{Question 3}
\end{document}



