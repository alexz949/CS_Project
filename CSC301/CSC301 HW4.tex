\documentclass{article}
\title{CSC301 HW4}
\author{Alex Zhang}
\date{Feb 2023}


\textwidth=16.00cm 
\textheight=22.00cm 
\topmargin=0.00cm
\oddsidemargin=0.00cm 
\evensidemargin=0.00cm 
\headheight=0cm 
\headsep=0.5cm
\textheight=610pt

\usepackage{amsthm,amssymb}

\newcommand{\bmat}[1]{\begin{bmatrix} #1 \end{bmatrix}}
\newcommand{\mat}[1]{\mathbf{#1}}
\newcommand{\mb}[1]{\mathbb{#1}}

\let\ds\displaystyle




\begin{document}
\maketitle


\section*{Question 1}
Based on the question, since $n$ is a power of 2, $n$ will always be even. The following is the pesudocode for the algorithm,
\begin{verbatim}
    function v = MEDIAN(A,B,n)
        if n == 1 then return (A[0]+B[0])/2
        
        m1 = (A[n/2] + A[n/2-1])/2
        m2 = (B[n/2] + B[n/2-1])/2
        if m1 == m2 
            then return m1
        if m1 < m2
            return MEDIAN(A[n/2,:], B, n/2)
        else
            return MEDIAN(A, B[n/2,:], n/2)
        end if
    end function
        
\end{verbatim}


\section*{Question 2}





\section*{Question 3}

\end{document}