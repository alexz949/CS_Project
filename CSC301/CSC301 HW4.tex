\documentclass{article}
\title{CSC301 HW4}
\author{Alex Zhang}
\date{Feb 2023}


\textwidth=16.00cm 
\textheight=22.00cm 
\topmargin=0.00cm
\oddsidemargin=0.00cm 
\evensidemargin=0.00cm 
\headheight=0cm 
\headsep=0.5cm
\textheight=610pt

\usepackage{amsthm,amssymb}

\newcommand{\bmat}[1]{\begin{bmatrix} #1 \end{bmatrix}}
\newcommand{\mat}[1]{\mathbf{#1}}
\newcommand{\mb}[1]{\mathbb{#1}}

\let\ds\displaystyle




\begin{document}
\maketitle


\section*{Question 1}
Based on the question, since $n$ is a power of 2, $n$ will always be even. The following is the pesudocode for the algorithm,
\begin{verbatim}
    function v = MEDIAN(A,B,n)
        if n == 1 then return max(A[0],B[0])
        
        m1 = (A[n/2])/2
        m2 = (B[n/2])/2
        if m1 == m2 
            then return m1
        if m1 < m2
            return MEDIAN(A[n/2,:], B, n/2)
        else
            return MEDIAN(A, B[n/2,:], n/2)
        end if
    end function
        
\end{verbatim}
    \subsection*{Proof of Correctness}
    In the base case where n = 1, the union array's larger median will either be the first
    element in $A$ or the first element in $B$, which the algorithm will return the bigger
    value between these two. So the base case is true.
    \\
    For inductive cases, we assume that the algorithm will work when the length is $n-1$,
    and we need to show that $n$ length will also work.
    

\section*{Question 2}





\section*{Question 3}

\end{document}