\documentclass{article}
\usepackage{amsmath}

\usepackage{amssymb}


\title{CSC301 HW2}
\author{Alex Zhang}
\date{Jan 2023}


\begin{document}

\maketitle

\section*{Question 1} 
Let algorithm computing $x^y$ that makes at most $O(n)$ calls to integer multiplication called EXPO.
The pseduocode is following below.
\begin{verbatim}
    function z = EXPO(x,y)
        if y = 0 then return 1
        z = EXPO(x, math.floor(y/2))
        z = z*z
        if y is even then
            return z
        else
            return z*x
        end if
    end function
\end{verbatim}
Proof:
\paragraph{Base Case: } $y = 0$ \\
The algorithm then returns $1$, which is true because $x^0 = 1$ for all $x$.
\paragraph{Induction: } Assume $y = n-1$ is true, we are going to prove that $y = n$ for $z = x^y $ is also true.\\
The recursion will get $z^\prime = x^{\lfloor y/2 \rfloor}$.
\paragraph{Case 1: } $y = n$ is even\\
The EXPO returns $z$.
If $y$ is even, then $\lfloor y/2 \rfloor = y/2 $, which 
$$z^\prime = x^{y/2}$$
squaring both sides,
$$(z^\prime)^2 = x^y$$
So $z  = (z^\prime)^2$ when $y$ is even. Just as what EXPO will return.

\paragraph{Case 2: } $y = n$ is odd\\
Then EXPO returns $z \cdot x$. In this case $\lfloor y/2 \rfloor = (y-1)/2$, which 
$$z^\prime = x^{y-1/2}$$
squaring both sides and multipling by $x$,
$$(z^\prime)^2 \cdot x = x^{y} $$
In this case $ (z^\prime)^2 = z$. So $(z^\prime)^2 \cdot x = z \cdot x$. This matched the EXPO returning value. $\blacksquare$
\\
\begin{enumerate}
\item{Integer multiplication}

Since we assume that $x$, $y$ are all n-bits numbers, so in each recursion, $y$ will chunk the last bit so there are total n times recursion calls.
In each call, it will perform a integer multiplication on $z$ it self, and sometimes possible calling multiplication for $x$ and $z$. So overall, there are $O(n)$ integer multiplication calls for the whole recursion process.

\item{Time Complexity}

We have showed that there are total $O(n)$ calls to integer multiplication, and for multiplication itself, th time Complexity will be $O(n^2)$, which overall, the time complexity for EXPO will be $O(n^3)$

\end{enumerate}


\section*{Question 2}
    \subsection*{(a)}
    If a number $ 1 \leq x < p$ for a given prime $p$ is its own multiplicative inverse, then
    $$x^2 \equiv 1 \mbox{ mod } p$$
    By definition, for some integer $k$: 
\begin{align}
    x^2 - 1 &= pk \nonumber \\
    (x+1)(x-1) &= pk \nonumber
\end{align}
We have two case for $k = 0$ and $k \neq 0$.

\paragraph{Case 1: }$k \neq 0$ \\
Since  $ 1 \leq x < p$ and $x$ is relative prime to $p$, so either $x+1 = p$ or $x-1 = p$. However, the second case will not happen since $x-1 < p-1$.
$x = p-1$ is one number that is its own multiplicative inverse.

\paragraph{Case 2:} $k = 0$\\
This time either $x-1 = 0$ or $x+1 = 0$, but $x+1 \geq 2$ with the given condition so $x - 1 = 0$ and $x= 1$.
\\
There are two numbers $1$ and $p-1$ which are their own inverse.

\subsection*{(b)}
$(p-1)! = (p-1)(p-2)(p-3)\dots 2 \cdot 1$, so
    $$(p-1)! \equiv -1 \mbox{ mod } p$$
    $$(p-1)(p-2)(p-3)\dots 2 \cdot 1 \equiv -1 \mbox{ mod } p$$
    $$-1 \cdot (p-2)(p-3)\dots 2 \cdot 1 \equiv -1 \mbox{ mod } p $$
We need to show that  $(p-2)(p-3) \ldots 2 \equiv 1 \mbox{ mod } p$.\\
Since for prime $p$, every number $1 \leq x < p$ is invertible modulo $p$, we have 
$$ax \equiv 1 \mbox{ mod } p $$
Taking $a$ mod $p$, $a \in \{1,2,\dots, p-2,p-1\}$ and since $1$ and $p-1$ are their own inverse,
so for each $x$, there is going to be a multiplicative inverse $a \in \{2,\dots,p-2\}$. Therefore, we can pairing up every element in $ \{2,\dots,p-2\}$ to make a multiplicative inverse respect to $p$.

$$-1 \cdot (p-2)(p-3)\dots 2 \cdot 1 \equiv -1 \mbox{ mod } p $$
will become
$$-1 \cdot 1 \cdot 1 \dots 1 \cdot 1 \equiv -1 \mbox{ mod } p $$
which is true.$\blacksquare$

\subsection*{(c)}
Suppose $N$ is a composite number, then there will be a factor $b > 1 \mbox{ and } b \leq N-1$. This indicates that $d$ divides $(N-1)!$, meaning that $d$ does not divide $(N-1)! + 1$, same as
$$(N-1)! \not\equiv -1 \mbox{ mod } N$$
So if $N$ is not prime, then  $(N-1)! \not\equiv -1 \mbox{ mod } N$. $\blacksquare$


\subsection*{(d)}
I think the problem is that Wilson's Theorem takes more time on doing the primality test. 
For example, the time complexity when $N$ is a n-bits number will be $O(2^nn^2)$ because the the maximum times of integer multiplication will be $2^{n-1}-1$ and each multiplication itself is $O(n^2)$. Whereas 
Fermat's Little Theorem only takes like $O(n^3)$ and maximum of $O(n^4)$ for doing $n$ times of check.

\section*{Question 3}
    \subsection*{(a)}
Based on the question, we know that $p$ is prime, and for encryption
integer $e$, and $e$ is relative prime to $s = p-1$. The following pseduocode is the method used for recovering $x$ for $0 \leq x < p$.
\begin{verbatim}
    function Decrypt(y,e,p)
        (d,a,b) = EXTEuclid(e,p-1)
        x = MODEXP(y,d,p)
        return x
    end function
\end{verbatim}
Proof: 
\paragraph{Case 1: } $x \not\equiv 0 \mbox{ mod } p$ \\
Since in this method, Extended Euclid's algorithm is used for getting integer $d$ where $de \equiv 1 \mbox{ mod } s$.
This means $de - 1 = ks$ for some inetger $k$.
$$x^{de} = x^{1+ks} = x\cdot x^{(p-1)k}$$
By Fermat's Little Theorem, if $x \not\equiv 0 \mbox{ mod } p$, then
$$x^{ks} \equiv x^{(p-1)k} \equiv 1^{k} \equiv 1 \mbox{ mod } p$$
Therefore 
$$x^{de} \equiv x\cdot x^{ks} \equiv 1\cdot x \equiv x \mbox{ mod } p $$

\paragraph{Case 2: }  $x \equiv 0 \mbox{ mod } p$ \\
Since for this RSA-like scheme, $ 0 \leq x < p$, then only possible number for $x$ is $0$.
This case the method will also be true since
$$0^{de} \equiv 0 \mbox{ mod } p$$
The correctness of this RSA-like scheme has benn proved. $\blacksquare$

\subsection*{Bit Complexity}

$\bullet$ Alice generates one n-bits prime $p$\\
$\bullet$ Alice computes $s = p-1$\\
$\bullet$ Alice chooses $e$  relatively prime to $s$\\
$\bullet$ Eve knows $e$ and $p$ and computes $s$ and $d \equiv e^{-1} \mbox{ mod } s$ \\
$\bullet$ Eve computes $x\equiv y^d \mbox{ mod } p$\\
\\
When $p$ is a n-bits prime number, the time complexity for this method is $O(n^3)$ overall. In step four, Eve
will use this method and use Extended Euclidea's Algorithm to get decryption integer $d$ which takes $O(n^3)$.
In step five, Eve then using the modular exponentiation to recover the original message $x$, which takes $O(n^3)$.
Overall, this method will take polynomial time to recover $x$.


\subsection*{(b)}
With given $s$ and public key $e$, we can use the Extended Euclidea's Algorithm to get the multiplicative inverse of $e$ where
$$de \equiv 1 \mbox{ mod } s$$
The time complexity for Extended Euclidea's Algorithm is $O(n^3)$ and this is polynomial time.
    





\end{document}