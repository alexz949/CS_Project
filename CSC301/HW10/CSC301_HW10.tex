\documentclass{article}
\title{CSC301 HW10}
\author{Alex Zhang}
\date{April 2023}
\textwidth=16.00cm 
\textheight=22.00cm 
\topmargin=0.00cm
\oddsidemargin=0.00cm 
\evensidemargin=0.00cm 
\headheight=0cm 
\headsep=0.5cm
\textheight=610pt
\usepackage{graphicx}
\usepackage{multicol}

\usepackage{relsize}




\graphicspath{ {./images/} }

\usepackage{latexsym,array,delarray,amsthm,amssymb,epsfig}
\usepackage{amsmath}
\usepackage{listings}
\lstset{
  basicstyle=\ttfamily,
  mathescape
}


\newcommand{\bmat}[1]{\begin{bmatrix} #1 \end{bmatrix}}
\newcommand{\mat}[1]{\mathbf{#1}}

\let\ds\displaystyle

\begin{document}
\maketitle
\section*{Question 1}

\section*{Question 2}
Since we proved that SAT $\rightarrow$ 3SAT in class, and both of them are NP-complete.
If we can prove that 3SAT $\rightarrow$ EXACT 4SAT, then SAT $\rightarrow$ EXACT 4SAT, which indicates that EXACT 4SAT
is also NP-complete. 
\subsection*{Define f}
\paragraph*{Case 1} Clause length 1\\

For the clause of length 1 $a_1$, we need to add three new "auxiliary" variables. For clause with length 1, we define 
f to be:


$$a_1 = (a_1\vee y_1 \vee y_2 \vee y_3) \wedge (a_1\vee \bar{y_1} \vee y_2 \vee y_3) \wedge (a_1\vee y_1 \vee \bar{y_2} \vee y_3) \wedge (a_1\vee y_1 \vee y_2 \vee \bar{y_3})$$
$$\wedge (a_1\vee \bar{y_1} \vee \bar{y_2} \vee y_3) \wedge (a_1\vee \bar{y_1} \vee y_2 \vee \bar{y_3}) \wedge (a_1\vee y_1 \vee \bar{y_2} \vee \bar{y_3}) \wedge (a_1\vee \bar{y_1} \vee \bar{y_2} \vee \bar{y_3}) $$

\paragraph*{Case 2} Clause length 2\\
In this case we need to add two more "auxiliary" variables, and we define f as:
$$(a_1 \vee a_2) = (a_1 \vee a_2 \vee y_1 \vee y_2) \wedge (a_1 \vee a_2 \vee y_1 \vee \bar{y_2}) \wedge (a_1 \vee a_2 \vee \bar{y_1} \vee y_2) \wedge (a_1 \vee a_2 \vee \bar{y_1} \vee \bar{y_2})$$

\paragraph*{Case 3} Clause length 3\\
We just need one more "auxiliary" variable. The f now will be:
$$(a_1 \vee a_2 \vee a_3) = (a_1 \vee a_2 \vee a_3 \vee y_1) \wedge (a_1 \vee a_2 \vee a_3 \vee \bar{y_1})$$


Above all, f will be in polynomial time since creating new auxiliary variables takes $O(m)$.

\subsection*{Define h}
h is the true assignment for EXACT 4SAT to solutions to 3SAT. Define h to ignore the truth assignemnt 
of auxiliary variables, keeping the truth assignment of the origonal variables. h is poly-time,
since we are just chops off at most 3 bits vector.


\subsection*{h(S) satisfies I}
Suppose not, then there are three cases.
\paragraph*{Case 1} Clause with length 1 is false \\
Then the false cluase can be transformed into:
$$a_k = (y_1 \vee y_2 \vee y_3) \wedge (\bar{y_1} \vee y_2 \vee y_3) \wedge (y_1 \vee \bar{y_2} \vee y_3) \wedge (y_1 \vee y_2 \vee \bar{y_3})$$
$$\wedge (\bar{y_1} \vee \bar{y_2} \vee y_3) \wedge (\bar{y_1} \vee y_2 \vee \bar{y_3}) \wedge (y_1 \vee \bar{y_2} \vee \bar{y_3}) \wedge (\bar{y_1} \vee \bar{y_2} \vee \bar{y_3}) $$
In this case, all three "auxiliary" variables need to be true. However, this will make the last clause to be false which leads to a contradiction.

\paragraph*{Case 2} Clause with length 2 is false \\
Then the false cluase can be transformed into:
$$(a_k \vee a_{k+1}) = (y_1 \vee y_2) \wedge (y_1 \vee \bar{y_2}) \wedge (\bar{y_1} \vee y_2) \wedge (\bar{y_1} \vee \bar{y_2})$$
Based on this string, we have to make both $y_1$ and $y_2$ to be true but this will still make the last clause be false. A contradiction happens.


\paragraph*{Case 3} Clause with length 3 is false \\
Then the false clause can be simplified into:
$$(a_k \vee a_{k+1} \vee a_{k+2}) = (y_1) \wedge (\bar{y_1})$$
and this implies that $y_1$ needs to be true, but this will lead a contradiction which $\bar{y_1}$ cannot.


\subsection*{I satisfies so that f(I)}
Suppose the origonal string is satisfied, then every clause regradless of length need to be true. 
\paragraph*{Case 1} Clause with length 1 \\
Based on the f in previous statement, it is clear that if $a_k$ is true, f will also be true since $a_k$ is in every clasue.

\paragraph*{Case 2} Clause with length 2 \\
Since $(a_k \vee a_{k+1})$ is true, adding two more auxiliary variables will also be true in each clause. Therefore length 2 will be true.

\paragraph*{Case 3} Clause with length 3 \\
Since $(a_k \vee a_{k+1} \vee a_{k+2})$ is true, adding an extra auxiliary variable will also be true without considering its boolean value.
The transformation will be true is the original stirng is true.

We can conclude that 3SAT $\rightarrow$ EXACT 4SAT. Based on the fact that SAT $\rightarrow$ 3SAT, we can use 
reduction compose to show that 
$$\text{SAT} \rightarrow \text{EXACT 4SAT}$$
So EXACT 4SAT is NP-complete. $\blacksquare$






\section*{Question 3}



\end{document}
