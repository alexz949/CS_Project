\documentclass{article}
\title{CSC301 HW10}
\author{Alex Zhang}
\date{April 2023}
\textwidth=16.00cm 
\textheight=22.00cm 
\topmargin=0.00cm
\oddsidemargin=0.00cm 
\evensidemargin=0.00cm 
\headheight=0cm 
\headsep=0.5cm
\textheight=610pt
\usepackage{graphicx}
\usepackage{multicol}

\usepackage{relsize}




\graphicspath{ {./images/} }

\usepackage{latexsym,array,delarray,amsthm,amssymb,epsfig}
\usepackage{amsmath}
\usepackage{listings}
\lstset{
  basicstyle=\ttfamily,
  mathescape
}


\newcommand{\bmat}[1]{\begin{bmatrix} #1 \end{bmatrix}}
\newcommand{\mat}[1]{\mathbf{#1}}

\let\ds\displaystyle

\begin{document}
\maketitle
\section*{Question 1}

Assume there is an $independent$ $set$ $S$ which in $G$. For any edge $e = (u,v)$. Only one of $u$, $v$ can be in
$S$. This means at least one of $u$, $v$ will be in $V-S$ which means any $e$ is adjacent to some vertex in  vertex cover $C$.
This indicates that for given $S$ in $G$. $V-S$ is the vertex cover.

Assume there is a $vertex$ $cover$ $C$ that is $V-S$. Taking all vertices that is not in $V-S$ and there will be no edges between any of two
vertices that are not in $V-S$. Therefore, the rest vertices become an $independet$ $set$ $S$.
\subsection*{(a)}
With given instance of given $G$ and $k$, we define f to change $k$ be $n-k$, where $n$ is the total number
of vertices.

Suppose we have already had an efficient algorithm to check whether $G$ has a $vertex$ $cover$ with size $\leq n-k$. 
Based on the relation of $S$ and $V-S$ we showed at the beginning, h is now just doing calculation of $n-(n-k)$.
Both f and h are in polynomial time because all about is counting the number of vertices in $G$.

Based on the efficient algorithm, if there exists a $vertex$ $cover$ with size $\leq n-k$, this implies that there is an $independent$ $set$ that has size
$\geq k$, since the size of $independent$ $set$ add size of $vertex$ $cover$ is the number of vertices in $G$.

If there does not exist a $vertex$ $cover$ with size $\leq n-k$, then there is no $independent$ $set$ with size greater than $k$.

This indicates that $vertex$ $cover$ problem can be reduced into $independent$ $set$ problem. $\blacksquare$

\subsection*{(b)}
This time we define f to change $l$ to $n-l$.

Assume we have an efficient algorithm that check whether $independent$ $set$ has size $\geq n-l$. We can also define h be calculating $n - (n-l)$. both 
f and h are in polynomial time because counting the number of vertices will not cost so much time.

Based on the algorithm, if it is true, then there exists an $independent$ $set$ with size $\geq n-l$. This means that there exists a $vertex$ $cover$ with size 
$\leq l$ based on h. If it is false, then there is no $independent$ $set$ with size $\geq n-1$. This also means there is no $vertex$ $cover$ with size 
$\leq l$ because if $S$ is an $independent$ $set$ , $V-S$ is a $vertex$ $cover$.

This shows that $independent$ $set$ problem can be reduced into $vertex$ $cover$ problem.$\blacksquare$

Overall $independent$ $set$ problem can be reduced to $vertex$ $cover$ problem and vise versa. If we just know one efficient algorithm, we can use it to solve 
two questions at the same time.



\section*{Question 2}
Since we proved that SAT $\rightarrow$ 3SAT in class, and both of them are NP-complete.
If we can prove that 3SAT $\rightarrow$ EXACT 4SAT, then EXACT 4SAT
is also NP-complete. 
\subsection*{Define f}
\paragraph*{Case 1} Clause length 1\\

For the clause of length 1 $a_1$, we need to add three new "auxiliary" variables. For clause with length 1, we define 
f to be:


$$a_1 = (a_1\vee y_1 \vee y_2 \vee y_3) \wedge (a_1\vee \bar{y_1} \vee y_2 \vee y_3) \wedge (a_1\vee y_1 \vee \bar{y_2} \vee y_3) \wedge (a_1\vee y_1 \vee y_2 \vee \bar{y_3})$$
$$\wedge (a_1\vee \bar{y_1} \vee \bar{y_2} \vee y_3) \wedge (a_1\vee \bar{y_1} \vee y_2 \vee \bar{y_3}) \wedge (a_1\vee y_1 \vee \bar{y_2} \vee \bar{y_3}) \wedge (a_1\vee \bar{y_1} \vee \bar{y_2} \vee \bar{y_3}) $$

\paragraph*{Case 2} Clause length 2\\
In this case we need to add two more "auxiliary" variables, and we define f as:
$$(a_1 \vee a_2) = (a_1 \vee a_2 \vee y_1 \vee y_2) \wedge (a_1 \vee a_2 \vee y_1 \vee \bar{y_2}) \wedge (a_1 \vee a_2 \vee \bar{y_1} \vee y_2) \wedge (a_1 \vee a_2 \vee \bar{y_1} \vee \bar{y_2})$$

\paragraph*{Case 3} Clause length 3\\
We just need one more "auxiliary" variable. The f now will be:
$$(a_1 \vee a_2 \vee a_3) = (a_1 \vee a_2 \vee a_3 \vee y_1) \wedge (a_1 \vee a_2 \vee a_3 \vee \bar{y_1})$$


Above all, f will be in polynomial time since creating new auxiliary variables takes $O(m)$.

\subsection*{Define h}
h is the true assignment for EXACT 4SAT to solutions to 3SAT. Define h to ignore the truth assignemnt 
of auxiliary variables, keeping the truth assignment of the origonal variables. h is poly-time,
since we are just chops off at most 3 bits vector.


\subsection*{h(S) satisfies I}
Suppose not, then there are three cases.
\paragraph*{Case 1} Clause with length 1 is false \\
Then the false cluase can be transformed into:
$$a_k = (y_1 \vee y_2 \vee y_3) \wedge (\bar{y_1} \vee y_2 \vee y_3) \wedge (y_1 \vee \bar{y_2} \vee y_3) \wedge (y_1 \vee y_2 \vee \bar{y_3})$$
$$\wedge (\bar{y_1} \vee \bar{y_2} \vee y_3) \wedge (\bar{y_1} \vee y_2 \vee \bar{y_3}) \wedge (y_1 \vee \bar{y_2} \vee \bar{y_3}) \wedge (\bar{y_1} \vee \bar{y_2} \vee \bar{y_3}) $$
In this case, all three "auxiliary" variables need to be true. However, this will make the last clause to be false which leads to a contradiction.

\paragraph*{Case 2} Clause with length 2 is false \\
Then the false cluase can be transformed into:
$$(a_k \vee a_{k+1}) = (y_1 \vee y_2) \wedge (y_1 \vee \bar{y_2}) \wedge (\bar{y_1} \vee y_2) \wedge (\bar{y_1} \vee \bar{y_2})$$
Based on this string, we have to make both $y_1$ and $y_2$ to be true but this will still make the last clause be false. A contradiction happens.


\paragraph*{Case 3} Clause with length 3 is false \\
Then the false clause can be simplified into:
$$(a_k \vee a_{k+1} \vee a_{k+2}) = (y_1) \wedge (\bar{y_1})$$
and this implies that $y_1$ needs to be true, but this will lead a contradiction which $\bar{y_1}$ cannot.


\subsection*{I satisfies so that f(I)}
Suppose the origonal string is satisfied, then every clause regradless of length need to be true. 
\paragraph*{Case 1} Clause with length 1 \\
Based on the f in previous statement, it is clear that if $a_k$ is true, f will also be true since $a_k$ is in every clasue.

\paragraph*{Case 2} Clause with length 2 \\
Since $(a_k \vee a_{k+1})$ is true, adding two more auxiliary variables will also be true in each clause. Therefore length 2 will be true.

\paragraph*{Case 3} Clause with length 3 \\
Since $(a_k \vee a_{k+1} \vee a_{k+2})$ is true, adding an extra auxiliary variable will also be true without considering its boolean value.
The transformation will be true is the original stirng is true.

We can conclude that 3SAT $\rightarrow$ EXACT 4SAT. Based on the fact that 3SAT is NP-complete,
so EXACT 4SAT is also NP-complete. $\blacksquare$






\section*{Question 3}



\end{document}
