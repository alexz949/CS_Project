\documentclass{article}
\usepackage{amsmath}
\usepackage{amssymb}
\usepackage{mathtools}
\title{CSC301 HW1}
\author{Alex Zhang}
\date{Januray 2023}

\begin{document}
\maketitle
\section{Question 1}
    \subsection{(a)}
    Based on the formula, $g(n) = \sum_{k=0}^{n} r^k = \frac{1-r^{n+1}}{1-r}$, for $r < 1$.
Then
    $$g(n) = \frac{1-r^{n+1}}{1-r} \leq \frac{1}{1-r} $$
since $r^{n+1} \geq 0$ when  $r \geq 0$.
We can then assume that $\exists c$ which $c = \frac{1}{1-r}$ becuase $r$ is a constant
and $\exists N$ where $N = 1$.
So,
$$g(n) \leq c \cdot 1$$
for all $n \geq N$, then 
$$g(n) = O(1)$$

    \subsection{(b)}
    When $r = 1$, $g(n)$ will be $1 + 1 + 1^2 + \dots  + 1^n$
$$g(n) =  1\cdot n + 1 = n + 1$$
So, there exists $c = 10$, $N = 1$ such that 
$$g(n) \leq  10 \cdot n$$
for all $n \geq N$, then 
$$g(n) = O(10n)$$
The time complexity for $g(n)$ when $r=1$ will be $O(n)$.
    \subsection{(c)}
    When $r > 1$, the formula for finite geometric series sum will be $\frac{r^{n+1}-1}{r-1}$.
Then, 
$$g(n) = \frac{r^{n+1}-1}{r-1} \leq \frac{r^{n+1}}{r-1}$$
We can assume that there exists a $c = \frac{1}{r-1}$, becuase $r$ is a constant so $1/r-1$ will also be a constant and exists $N = 1$.
This satisfies that 
$$g(n) \leq c \cdot r^{n+1}$$
for all $n \geq N$, then 
$$g(n) = O(r^{n+1})$$

The time complexity for $g(n)$ when $r > 1$ will be $O(r^n)$.



\section{Question 2}
    \subsection{(a)}
    \begin{verbatim}
    public static void main(String[] args) {
        int[] test = new int[2];
        test = divide(0b1011, 0b10);
        System.out.println("the quotient will be: " + test[0]);
        System.out.println("the remainder will be: " + test[1]);
    }

    static int[] divide(int x, int y){
        int[] ans = new int[2];
        if(x == 0)
            return ans;
        ans = divide((int)Math.floor(x / 2), y);
        ans[0] *= 2;
        ans[1] *= 2;

        if(x % 2 == 1)
            ans[1] += 1;
        if(ans[1] >= y){
            ans[1] -= y;
            ans[0] += 1;
        }
        return ans;
    }
    \end{verbatim}
    My input for $x$ is $1011$ and $y$ is $10$ in binary.\\\\\\
    For line
    \begin{verbatim}
        if(x == 0)
            return ans;
    \end{verbatim}
    This happens in the last recusion call when $x = 0, y = 10$, and it will return the array.\\
    For lines
    \begin{verbatim}
        ans = divide((int)Math.floor(x / 2), y);
        ans[0] *= 2;
        ans[1] *= 2;
    \end{verbatim}
    The first line is doing the recusion and the rest will execute in every recursion.
    This condition
    \begin{verbatim}
        if(x % 2 == 1)
            ans[1] += 1;
    \end{verbatim}
    will be true when $x = 1, 101, 1011$ since they are all odd numbers.\\
    The last condition
    \begin{verbatim}
        if(ans[1] >= y){
            ans[1] -= y;
            ans[0] += 1;
        }
    \end{verbatim}
    will execute when $x = 10$ and $1011$ which the remainder will be $2$ and $3$ respectively.

    \subsection{(b)}

The base case will be $x = 0$ and $y$ is an abitary number.
Base case is true since the algorithm will return $q = 0, r = 0$, which satisfies
$$0 = 0 \cdot y + 0 \mbox{ with } 0 \leq r < y $$
\\
For induction case, assume that $x = n$ and given an abitary $y$, which satisfies
$$n = q \cdot y + r \mbox{ with } 0\leq r < y $$
Then 
$$\lfloor n/2 \rfloor = q^\prime \cdot y + r^\prime $$
\\
\paragraph{Case 1: $n$ is even}

Then $\lfloor n/2 \rfloor = n/2$.


\begin{align}
    n/2  &= q^\prime \cdot y + r^\prime \nonumber \\
    n &=  2q^\prime \cdot y + 2r^\prime \nonumber
\end{align}
If $2r^\prime < y$, then $2q^\prime = q$, $2r^\prime = r$.
$$n = q \cdot y + r \mbox{ with } 0 \leq r < y$$ which is true.
\\
If $2r^\prime \geq y$, for the algorithm, 
$$n = (2q^\prime+1)y + (2r^\prime-y)$$
So $2q^\prime + 1 = q$ and $2r^\prime - y = r$ which is also true since
$$ 0 \leq r^\prime < y $$$$  0 \leq 2r^\prime < 2y $$$$  0 \leq 2r^\prime - y < y  $$
We can get 
$$n = q \cdot y + r \mbox{ with } 0 \leq r < y$$




\paragraph{Case 2: $n$ is odd}

Then $\lfloor n/2 \rfloor = (n-1)/2$.
\begin{align}
    (n-1)/2 &= q^\prime \cdot y + r^\prime \nonumber \\
    n &=  2q^\prime \cdot y + 2r^\prime + 1 \nonumber
\end{align}
For $2r^\prime + 1 < y$ and $y\leq 2r^\prime + 1 < 2y$, without loss of generality, the proving is the same in case 1, which is true.
\\
For $2r^\prime + 1 = 2y$, it is impossible since  $2r^\prime + 1$ will always be an odd number but $2y$ will always be an even number.
\\
Overall, the indcution shows that this algorithm is correct.$\blacksquare$

    \subsection{(c)}
    Assume that input $x$, $y$ are n-bits numbers, then this algorithm takes $n$ times of recursion calls becuase each time the number of bit decrease by one.
    For each recursion call, for the worst case, every call will go into 
    \begin{verbatim}
        if(ans[1] >= y){
            ans[1] -= y;
            ans[0] += 1;
        }
    \end{verbatim}
    condition, which addition is performed. This required linear time complexity.
    \\
    Overall, there are $n$ times of recursion calls for n-bits number, and for each call, there is going to be an addition performing linear complexity.
    So the total time complexity for this algorithm is $O(n^2)$.

    
    

    

\section{Question 3}
By definition, $x \equiv y \mbox{ mod } N $ if and only if $N$ divides $x-y$.
In question since $x \equiv x^\prime \mbox{ mod } N$, and $y \equiv y^\prime \mbox{ mod } N $, equivalently,
$$x - x^\prime = N \cdot k$$
$$y - y^\prime = N \cdot l$$
for $k$, $l$ $\in \mathbb{Z}$
\\
Then $x = N \cdot k + x^\prime$, $y = N \cdot l + y^\prime$, and 
\begin{align}
    xy &= (N \cdot k + x^\prime) (N \cdot l + y^\prime) \nonumber \\
xy &= N^2 k l + Nk y^\prime + Nlx^\prime + x^\prime y^\prime \nonumber \\
xy &= N(Nkl + k y^\prime + lx^\prime) + x^\prime y^\prime \nonumber \\
xy - x^\prime y^\prime &= N(Nkl + k y^\prime + lx^\prime) \nonumber
\end{align}
By definition, let $m \in \mathbb{Z}$ and $m = (Nkl + k y^\prime + lx^\prime)$ since integer is closed under addition.

$$xy - x^\prime y^\prime = N \cdot m$$
which by definition again,

$$xy \equiv x^\prime y^\prime \mbox{ mod } N $$
The substitution rule for modular multiplication is true.$\blacksquare$


    
\end{document}