\documentclass{article}

\title{CSC301 HW3}
\author{Alex Zhang}
\date{Jan 2023}




\usepackage{amsthm,amssymb}
\usepackage{amsmath}
\textwidth=16.00cm 
\textheight=22.00cm 
\topmargin=0.00cm
\oddsidemargin=0.00cm 
\evensidemargin=0.00cm 
\headheight=0cm 
\headsep=0.5cm
\textheight=610pt



\begin{document}
\maketitle


\section*{Question 1}

    \subsection*{(a)}
    Since $n \geq n-1 \geq n-2 \geq n-3 \geq \dots $, so that $n \cdot n \geq n \cdot  (n-1)$. We can then apply this inequality
    with more numbers which  
    $$n \cdot (n-1) \cdot (n-2) \cdot (n-3) \dots  1 \leq n\cdot n \dots n$$
    This inequality holds true because each element on the left side is smaller or equal to elements on the right side.
    Simplifying the inequality,
    $$n! \leq n^n$$
    which shows that it is true.$\blacksquare$
    \subsection*{(b)}
    Takes the $\log_{n/2}$ for $(n/2)^{n/2}$, which equals 
    $$\log_{n/2}(n/2)^{n/2} = n/2\log_{n/2}(n/2) = n/2$$
    Takes the $\log_{n/2}$ for $n$ factorial. This equals
    $$\log_{n/2}(n!) = \sum_{i=0}^{n-1}\log_{n/2}(n-i)$$
    Given a log function $\log_ab$, as long as $b \geq a$, $\log_ab \geq 1$. Expanding $\sum_{i=0}^{n-1}\log_{n/2}(n-i)$:
    $$\sum_{i=0}^{n-1}\log_{n/2}(n-i) = \log_{n/2}(n) + \log_{n/2}(n-1) + \dots + \log_{n/2}1$$
    We can get that all elements before $\log_{n/2}(n/2-1)$ is larger or equal to 1, and there are total $n/2 + 1$
    elements before $n/2-1$ in this summation. Therefore, we can obtain
    the following inequality:
    $$\sum_{i=0}^{n-1}\log_{n/2}(n-i) = \log_{n/2}(n) + \log_{n/2}(n-1) + \dots + \log_{n/2}1 \geq n/2 + 1$$
    Which is the same as,
    $$\log_{n/2}(n!) \geq n/2 + 1 \geq n/2 = \log_{n/2}(n/2)^{n/2} $$
    Exponentiates both sides,
    $$n! \geq (n/2)^{n/2}$$
    Just as the prompt.$\blacksquare$


    \subsection*{(c)}
    From question $(a)$ and $(b)$, we can get the inequality,
    $$n^n \geq n! \geq (n/2)^{n/2}$$
    Takes the log for all of them,
    $$n\log n \geq \log(n!) \geq (n/2)\log(n/2) $$

    \paragraph{Case 1: } Big-Oh\\
    Let $f(n) = \log(n!)$ and $ c \cdot g(n) = c \cdot n \log n$. By definition, Since
    $$\log(n!) \leq n\log n$$
    We can let $c = 1$ and $N = 1$, and plug in the number into inequality,
    $$f(n) = \log(n!) \leq n\log n = g(n)$$
    for all $n \geq N$. Therefore,
    $$\log(n!) = O(n\log n)$$

    \paragraph{Case 2: } Big-Omega\\
    Since $\log(n!) \geq (n/2)\log(n/2)$, we can do some transformation on the right hand side,
    $$\log(n!) \geq (n/2)\log n - (n/2)\log 2$$
    When $n \geq 4$, $n/4 \log n \geq n/2$ and substitudes $n/2 \log 2$ with
    $n/4 \log n$, we can get:
    $$\log(n!) \geq (n/2)\log n - n/4\log n = n/4\log n \mbox{ when } n \geq 4$$
    By definition, let $f(n) = \log(n!)$, and $c \cdot g(n) = c \cdot n/4 \log n $.
    We can assume that for $c = 4$ and $N = 4$, the inequality 
    $$\log(n!) \geq n\log n$$
    holds.\\
    So for all $n \geq N$, then 
    $$\log(n!) = \Omega(n\log n)$$
    Overall, if $\log(n!) = O(n\log n)$, and $\log(n!) = \Omega(n\log n)$, then
    $$\log(n!) = \Theta(n \log n)$$
    $\blacksquare$








  
\section*{Question 2}
Assume that the time complexity of function MULTIPLY for n-bits number
is $T(n)$. In each recursion calls, the 
input is seperated into 3 parts(line 4 and 5), and following 5 times of recursions
(showed in line $6$ to $10$). For partitioning, the time complexity is $O(n)$ since we assume the division by 3 is linear
and the multiplication of 2 is just adding zeros behind. For the rest part, since number addtion and subtraction
all cost linear time and division by 3 is also $O(n)$ as well as for the returning value. We can say that 
for each recursion call, the time complexity is $O(n)$ and we can draw the following formula,
$$T(n) = 5T(n/3) + O(n)$$
Also the base case $T(1)$ when $n = 1$ also has time complexity $O(1)$ becuase multiplying two 1-bit number costs
in constant time. Based on the information above, we are able to use master
theorem to get the time complexity for MULTIPLY since $a = 5$, $b = 3$, and $d = 1$.\\
Using master theorem, we get
$$\log_b a = \log_3 5 \approx 1.46 > 1 = d$$
So we will apply third case for master theorem which 
$$T(n) = O(n^{\log_b a}) = O(n^{\log_3 5}) = O(n^{1.46})$$
The time complexity for MULTIPLY is $O(n^{1.46})$.





\section*{Question 3}
\subsection*{(a) $T(n) = T(n/2) + O(\log n)$}
We can use back-substitution to solve the recursion.
    \begin{align}
        T(n) & = T(n/2) + O(\log n) \nonumber \\
        T(n/2) & = T(n/4) +  O(\log n/2) \nonumber \\
        T(n/4) &= T(n/8) + O(\log n/4) \nonumber \\
        \vdots \nonumber \\
        T(1) &= O(1) \nonumber
    \end{align}
Where for a number $k$,
$$T(n) = T(n/2^k) + \sum_{i=0}^{k-1}\cdot c \cdot  \log n/2^i$$
$$T(n) = T(n/2^k) + k \cdot   c \log n + c_1$$
Where $c_1$ is the sum of constant after using log division property.\\
In back-substitution, there are total $L$ level which $n/2^L = 1 \Rightarrow L = \log_2 n$.
Then, 
    \begin{align}
        T(n) &= T(1) + \log n \cdot c^\prime \log n + c_1 \nonumber \\
        T(n) &= c^\prime (\log n)^2 +  c + c_1 \nonumber 
    \end{align}
    Because $c^\prime  (\log n)^2$ is the dominating term in this equation and $c$, $c_1$ is a constant
    , we can get the time complexity for $T(n) = O((\log n)^2)$

\subsection*{(b) $T(n) = 7\cdot T(n/2) + O(n^2)$}
In this case, since $a = 7$, $b = 2$, and $d = 2$, and $T(1) = O(1)$ we can use master theorem to solve the recurrence.
$$\log_b a = \log_2 7 \approx 2.807 > 2 = d $$
We will go in case 3 of master theorem then,
$$T(n) = O(n^{\log_b a}) = O(n^{2.807})$$

\subsection*{(c) $T(n) = T(n-1) + O(n)$}
Back-substitution can be used for solving,
    \begin{align}
        T(n) & = T(n-1) + O(n) \nonumber \\
        T(n-1) & = T(n-2) + O(n-1) \nonumber \\
        T(n-2) & = T(n-3) + O(n-2) \nonumber \\
        \vdots \nonumber \\
        T(1) &= O(1) \nonumber
    \end{align}
For a number $k$, $T(n)$ can be expressed as 
$$T(n) = T(n-k) + c^\prime \cdot \sum_{i=0}^{k-1}(n-i)$$
For leave level $L$, $L$ satisfies $n-L = 1 \Rightarrow L = n-1$ and plug in $k = L$,
    \begin{align}
        T(n) &= T(1) + c^\prime  \cdot \sum_{i=0}^{n-2}(n-i) \nonumber \\
        T(n) &= c \cdot 1 + c^\prime \cdot (n+2)(n-1)/2 \nonumber \\
        T(n) &= c + c^\prime n^2 / 2 + c^\prime n/2 - c^\prime \nonumber 
    \end{align}
Since $c^\prime n^2 / 2$ is the dominating term and $c$ , 1 is a constant, and let $c^\prime  = 2$, we can get the time complexity $T(n) = O(n^2)$.
\subsection*{(d) $T(n) = 2\cdot T(n/2) + O(n)$}
Since $a = 2$, $b = 2$, and $d = 1$ with base case $T(1) = O(1)$, we can use master theorem solving the recurrence.
$$\log_b a = \log_2 2 = 1 = d$$
It goes into second case, where $T(n) = O(n^d\log n)$ and plugging in numbers.
$$T(n) = O(n\log n)$$

\subsection*{(e) $T(n) = 2\cdot T(n-1) + O(1)$ }
We can still use the substitution method.
    \begin{align}
        T(n) &= 2\cdot T(n-1) + O(1) \nonumber \\
        T(n-1) &= 2\cdot T(n-2) + O(1) \nonumber \\
        T(n-2) &=  2\cdot T(n-3) + O(1) \nonumber \\
        \vdots \nonumber \\
        T(1) &= O(1) \nonumber
    \end{align}
Given a number $k$, uses $n-k$ to represent $T(n)$ which,
$$T(n) = 2^k \cdot T(n-k) + c \cdot \sum_{i=0}^{k-1}2^i$$
In this case at leave level $L$,
$$T(n) = 2^L \cdot T(n-L) + c\cdot \sum_{i = 0}^{L-1}2^i$$
which $n-L = 1$, $L = n-1$,
    \begin{align}
        T(n) &= 2^{n-1}T(1) + c\cdot \sum_{i=0}^{n-2}2^i \nonumber \\
        T(n) &= 2^{n-1} \cdot c^\prime  + c\cdot  \frac{2^{n-1} - 1}{1}\nonumber \\
        T(n) &= 2^{n-1} \cdot (c^\prime + c)  + c \nonumber \\
        T(n) &= \frac{c^\prime+c}{2} 2^{n} + c \nonumber
    \end{align}
In this case since $c^\prime$ and $c$ are both constant, the time complexity for $T(n)$ will be 
$$T(n) = O(2^n)$$













\end{document}