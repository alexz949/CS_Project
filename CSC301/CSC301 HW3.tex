\documentclass{article}

\title{CSC301 HW3}
\author{Alex Zhang}
\date{Jan 2023}




\usepackage{amsthm,amssymb}
\textwidth=16.00cm 
\textheight=22.00cm 
\topmargin=0.00cm
\oddsidemargin=0.00cm 
\evensidemargin=0.00cm 
\headheight=0cm 
\headsep=0.5cm
\textheight=610pt



\begin{document}
\maketitle


\section*{Question 1}

    \subsection*{(a)}
    Since $n \geq n-1 \geq n-2 \geq n-3 \geq \dots $, so that $n \cdot n \geq n \cdot  (n-1)$. We can then apply this inequality
    with more numbers which  
    $$n \cdot (n-1) \cdot (n-2) \cdot (n-3) \dots  1 \leq n\cdot n \dots n$$
    This inequality holds true because each element on the left side is smaller than elements on the right side.
    Simplifying the inequality,
    $$n! \leq n^n$$
    which shows that it is true.$\blacksquare$
    \subsection*{(b)}
    Takes the $\log_{n/2}$ for $(n/2)^{n/2}$, which equals 
    $$\log_{n/2}(n/2)^{n/2} = n/2\log_{n/2}(n/2) = n/2$$
    Takesthe $\log_{n/2}$ for $n$ factorial. This equals
    $$\log_{n/2}(n!) = \sum_{i=0}^{n-1}\log_{n/2}(n-i)$$
    Given a log function $\log_ab$, as long as $b \geq a$, $\log_ab \geq 1$. Expanding $\sum_{i=0}^{n-1}\log_{n/2}(n-i)$:
    $$\sum_{i=0}^{n-1}\log_{n/2}(n-i) = \log_{n/2}(n) + \log_{n/2}(n-1) + \dots + \log_{n/2}1$$
    We can get that all elements before $\log_{n/2}(n/2-1)$ is larger or equal to 1, and there are total $n/2 + 1$
    elements before $n/2-1$ in this summation. Therefore, we can obtain
    the following inequality:
    $$\sum_{i=0}^{n-1}\log_{n/2}(n-i) = \log_{n/2}(n) + \log_{n/2}(n-1) + \dots + \log_{n/2}1 \geq n/2 + 1$$
    Which is the same as,
    $$\log_{n/2}(n!) \geq n/2 + 1 \geq n/2 = \log_{n/2}(n/2)^{n/2} $$
    Exponentiates both sides,
    $$n! \geq (n/2)^{n/2}$$
    Just as the prompt.$\blacksquare$


    \subsection*{(c)}
    From question $(a)$ and $(b)$, we can get the inequality,
    $$n^n \geq n! \geq (n/2)^{n/2}$$
    Takes the log for all of them,
    $$n\log n \geq \log(n!) \geq (n/2)\log(n/2) $$

    \paragraph{Case 1: } Big-Oh\\
    Let $f(n) = \log(n!)$ and $ c \cdot g(n) = c \cdot n \log n$. By definition, Since
    $$\log(n!) \leq n\log n$$
    We can let $c = 1$ and $N = 1$, and plug in the number into inequality,
    $$f(n) = \log(n!) \leq n\log n = g(n)$$
    for all $n \geq N$. Therefore,
    $$\log(n!) = O(n\log n)$$

    \paragraph{Case 2: } Big-Omega\\
    Since $\log(n!) \geq (n/2)\log(n/2)$, we can do some transformation on the right hand side,
    $$\log(n!) \geq (n/2)\log n - (n/2)\log 2$$
    When $n \geq 4$, $n/4 \log n \geq n/2$ and substitudes $n/2 \log 2$ with
    $n/4 \log n$, we can get:
    $$\log(n!) \geq (n/2)\log n - n/4\log n = n/4\log n \mbox{ when } n \geq 4$$
    By definition, let $f(n) = \log(n!)$, and $c \cdot g(n) = c \cdot n/4 \log n $.
    We can assume that for $c = 4$ and $N = 4$, the inequality 
    $$\log(n!) \geq n\log n$$
    holds.\\
    So for all $n \geq N$, then 
    $$\log(n!) = \Omega(n\log n)$$
    Overall, if $\log(n!) = O(n\log n)$, and $\log(n!) = \Omega(n\log n)$, then
    $$\log(n!) = \Theta(n \log n)$$
    $\blacksquare$








  
\section*{Question 2}

\section*{Question 3}




\end{document}