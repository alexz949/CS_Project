\documentclass{article}
\title{CSC355 PS1}
\author{Alex Zhang}
\date{Jan 2024}
\textwidth=16.00cm 
\textheight=22.00cm 
\topmargin=0.00cm
\oddsidemargin=0.00cm 
\evensidemargin=0.00cm 
\headheight=0cm 
\headsep=0.5cm
\textheight=610pt
\usepackage{graphicx}
\usepackage{multicol}

\graphicspath{ {./images/} }

\usepackage{latexsym,array,delarray,amsthm,amssymb,epsfig}
\usepackage{amsmath}
\usepackage{listings}
\lstset{
  basicstyle=\ttfamily,
  mathescape
}

\newcommand{\bmat}[1]{\begin{bmatrix} #1 \end{bmatrix}}
\newcommand{\mat}[1]{\mathbf{#1}}

\let\ds\displaystyle

\begin{document}
\maketitle

\section{Problem 1}
\subsection*{1.}
Given $f(x) = (1-x) \ln x$, $P_3(x)$ at $x_0 = 1$ will be
\begin{align}
    P_3(x) &= f(x_0) + f^\prime (x_0)(x-x_0) + \frac{f^2(x_0)}{2}(x-x_0)^2 + \frac{f^3(x_0)}{6}(x-x_0)^3 \nonumber \\
    &= 0 + 0/1(x-x_0) + -2/2(x-x_0)^2 + 3/6(x-x_0)^3 \nonumber \\ 
    &= -(x-1)^2 + \frac{1}{2}(x-1)^3 \nonumber
\end{align}
residual function $R_3(x)$ will be 
$$R_3(x) = \frac{f^4(c(x))}{24}(x-1)^4 = -\frac{2c(x) + 6}{24c(x)^4}(x-1)^4$$


\subsection*{2.}
Plug in x = 0.5, $P_3(0.5)$ equals,
$$P_3(0.5) = -(0.5)^2 + \frac{1}{2}(-0.5)^3 = - 0.3125$$
The residual function $R_3(0.5)$ will be $-\frac{0.0625}{24} \cdot \frac{2c(0.5) + 6}{c(0.5)^4} = -0.0026 \cdot \frac{2c(0.5) + 6}{c(0.5)^4}$.
Because $c(x)$ is bound by $0.5$ and $1$, we can use the Extreme Value Theorem to find an absolute maximum value given that interval.
\\
\\
Taking the first derivative of $\frac{2c(0.5) + 6}{c(0.5)^4}$ equals,
$$-\frac{6c(x)+24}{c(x)^5}$$
We can see that when $c(x) = -4$, there is one extreme value, but it is not in the interval.
We then calculate two end points.
\\
\\
$$\frac{2\cdot 0.5 + 6}{0.0625} = 112$$
$$\frac{2\cdot 1 + 6}{1} = 8$$
So the maximum value is 112, which means $|R_3(0.5)|$ is bounded by,
$$|R_3(0.5)| \leq \frac{0.0625}{24} \cdot 112 = 0.2917$$
which is the upper bound for $|f(0.5) - P_3(0.5)|$
\\
The actual error is 0.0341, which I think there is a huge difference.

\subsection*{3.}
Finding the bound of error $|f(x) - P_3(x)|$ for any $x \in [0.5,1.5]$ is the same as finding bound for $|R_3(x)|$ in the same interval.
We know that the choice of $c(x)$ in residual function is bounded by $x$ and $x_0$, which in this case we can just assume $c(x)$ is in interval $[0.5,1.5]$ given $x_0=1$.
\\
Based on the previous problem, the interval does not reach $-4$, we can still calculate end points to get maximum value of that fourth derivative.
$$\frac{2\cdot 0.5 + 6}{0.0625} = 112$$
$$\frac{2\cdot 1.5 + 6}{1.5^4} = 1.7778$$
For the maximum value of $\frac{(x-1)^4}{24}$ for $x \in [0.5,1.5]$, we can still use the extreme value theorem.
Compared the end points and point where derivative is zero, we found out the maximum value of $\frac{(x-1)^4}{24}$ is $0.0625/24$.
The bound for the error $|f(x) - P_3(x)|$ for any $x \in [0.5, 1.5]$ is $\frac{0.0625}{24} * 112 = 0.2917$.
\subsection*{4.}
The integral value of $\int_{0.5}^{1.5}P_3(x)dx$ will be,
\begin{align}
    \int_{0.5}^{1.5}P_3(x)dx &= \int_{0.5}^{1.5}-(x-1)^2 + 1/2(x-1^3)dx \nonumber \\
    &= \int_{0.5}^{1.5} -x^2 + 2x - 1 + \frac{x^3}{2} - \frac{3x^2}{2} + \frac{3x}{2} - \frac{1}{2}dx \nonumber \\
    &= -\frac{x^3}{3} + x^2 -x + \frac{x^4}{8} - \frac{x^3}{2} + \frac{3x^2}{4} - \frac{x}{2}\Bigg|_{0.5}^{1.5} \nonumber\\
    &= - \frac{1}{12} \nonumber
\end{align}



\subsection*{5.}
The integral $\int_{0.5}^{1.5}|R_3(x)|dx$ can be rewritten into
$$\int_{0.5}^{1.5}|R_3(x)|dx = \frac{1}{24}\int_{0.5}^{1.5}(x-1)^4 \frac{2c(x)+6}{c(x)^4}dx$$
We know that $\frac{2c(x)+6}{c(x)^4}$ in this interval is smaller than $112$.
This integral will be always smaller than
$$\frac{112}{24}\int_{0.5}^{1.5}(x-1)^4dx$$
Calculating this integral, we get $112/24 * 0.0125 = 0.05833$, which is the upper bound for the absolute error in 4.
\\
The actual error for that two expression is $|0.08802 - 0.83333| = 0.004687$. I think the upper bound for the absolute error is relatively large compared to the actual integral error.

\section*{Problem 2}

There are total two files for this problem.
One is called findinterval.m which is the function script.
The second one is called test.m which is the test script.
For function findinterval it is a function that does what the description says.
It will terminate until two end points have different signs or either function value or input value exceed maximum real number in MATLAB.
More descriptions can be found typing help findinterval.
\\
The test.m script just includes two start points which the first one is the successful case and the second one is unsuccessful case.
It will print two end points value if success.




\end{document}