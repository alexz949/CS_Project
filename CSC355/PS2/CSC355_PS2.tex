\documentclass{article}
\title{CSC355 PS2}
\author{Alex Zhang}
\date{Feb 2024}
\textwidth=16.00cm 
\textheight=22.00cm 
\topmargin=0.00cm
\oddsidemargin=0.00cm 
\evensidemargin=0.00cm 
\headheight=0cm 
\headsep=0.5cm
\textheight=610pt
\usepackage{graphicx}
\usepackage{multicol}


\graphicspath{ {./images/} }

\usepackage{latexsym,array,delarray,amsthm,amssymb,epsfig}
\usepackage{amsmath}
\usepackage{listings}
\lstset{
  basicstyle=\ttfamily,
  mathescape
}

\newcommand{\bmat}[1]{\begin{bmatrix} #1 \end{bmatrix}}
\newcommand{\mat}[1]{\mathbf{#1}}

\let\ds\displaystyle

\begin{document}
\maketitle

\section*{Problem 1}
\subsection*{1.}
The derivative of newton's update function $\phi^\prime(x)$ will be 
$$\phi^\prime(x) = 1 - \frac{f^\prime(x)f^\prime(x) - f(x) f''(x)}{[f^\prime (x)]^2} = \frac{f(x)f''(x)}{[f^\prime (x)]^2}$$ 
Since $\phi(x)$ is a newton's update function and sequence ${p_n}$ converges to root $p$, we can get $p_{n+1} = \phi(p_n)$.
If we plug in $p$ in update function, the second term will be 0 because of the root. $\phi(p) = p - 0 = p$.
So $p_{n+1} - p = \phi(p_n) - \phi(p)$.
\\
By Mean Value Theorem, $\phi(p_n) - \phi(p) = \phi^\prime(c(x))(p_n - p)$, where $c(x)$ is between $p_n$ and $p$.
Since $\{p_n\}^{\infty}_{n=0}$ converges to $p$, we also have $\{c(x)\}^\infty_{n=0}$ converges to p, which 
$$\lim_{n \rightarrow \infty}\phi^\prime(c(x)) = \phi^\prime(p)$$
\\
The left hand side of the equation we try to get can be transformed into:
\begin{align}
  \lim_{n \rightarrow \infty} \frac{| \phi^\prime(c(x))(p_n - p) |}{| p_n - p |} &= \lim_{n \rightarrow \infty} |\phi^\prime(c(x))|\frac{|p_n - p |}{| p_n - p |}  \nonumber \\
  \lim_{n \rightarrow \infty} |\phi^\prime(c(x))| &= |\phi^\prime(p) |\nonumber 
\end{align} 
Because $p$ is the root, $f(p) = 0$. We can plug in this number into $\phi^\prime(x)$ and get $\phi^\prime(p) = 0 $.
This means $|\phi^\prime(p)| = \phi^\prime(p)$. 
We then can show 
$$\lim_{n \rightarrow \infty} \frac{| p_{n +1}- p |}{| p_n - p |} = |\phi^\prime(p)| = \phi^\prime(p)$$

\subsection*{2.}
If newton's method converges to this root, we can create a Newton's update function $\phi(x) = x - \frac{f(x)}{f^\prime(x)}$.
Taking the derivative of this update function $\phi^\prime(x) = \frac{f(x)f''(x)}{[f^\prime(x)]^2}$ from previous problem.
Since now we know the representation of $f(x)$, we can try to substitute to the derivative of update function.
\\
$$f^\prime(x) = 2(x-p)g(x) + (x-p)^2g^\prime(x)$$
$$f''(x) = 2g(x) + 2(x-p)g^\prime(x) + 2(x-p)g^\prime(x) + (x-p)^2g''(x)$$
substitute these into the update function,
\begin{align}
  \phi^\prime(x) &= \frac{(x-p)^2g(x)( 2g(x) + 2(x-p)g^\prime(x) + 2(x-p)g^\prime(x) + (x-p)^2g''(x) )}{[2(x-p)g(x) + (x-p)^2g^\prime(x)]^2} \nonumber \\
  \phi^\prime(x) &= \frac{2g(x)^2 + 4(x-p)g^\prime(x)g(x) + (x-p)^2g''(x)g(x)}{(x-p)^2g^\prime(x)^2 + 4(x-p)g^\prime(x)g(x) + 4g(x)^2}\nonumber
\end{align}
Applying $x = p$,
$$\phi^\prime(p) = \frac{2g(x)^2}{4g(x)^2} = \frac{1}{2}$$
Because the derivative of update function is not $0$. The taylor theorem implies
$$\frac{\phi(x) - \phi(p)}{x -p} = \phi^\prime(p) = \frac{1}{2}$$
Assume Newton's method converges, we can get the same series in previous problem that also converges to $p$. Taking the limits,
$$\lim_{n \rightarrow \infty}\frac{|\phi(p_n) - \phi(p)|}{|p_n-p|}=\lim_{n \rightarrow \infty}\frac{|p_{n+1} - p|}{|p_n-p|} = \lim_{n \rightarrow \infty}|\phi^\prime(c(x))|$$
where $c(x)$ is between $p_n$ and $p$. 
\\
Because series $\{p_n\}^\infty_{n=0}$ converges to $p$, $c(x)$ also converges to $p$.
$$\lim_{n \rightarrow \infty}|\phi^\prime(c(x))| = |\phi^\prime(p)| = \frac{1}{2}$$.
This means Newton's method is linear convergent where $\alpha = 1$ and $\lambda = 1/2$, with asymptotic error constant 1/2.


\section*{Problem 2}
MATLAB script called newbis.m performs the required steps

\section*{Problem 3}
MATLAB script my$\_$fzero.m performs this combination.

\section*{Problem 4}
Function my$\_$fzero.m will print the information.
\\
\\
\\
\\
\\


\section*{Problem 5}

fzero with 0.8:

\begin{verbatim}
  >> p = fzero(ftest, 0.8,optimset('display','iter'));
 
  Search for an interval around 0.8 containing a sign change:
   Func-count    a          f(a)             b          f(b)        Procedure
      1             0.8      -0.76032           0.8      -0.76032   initial interval
      3        0.777373     -0.904332      0.822627      -0.63032   search
      5           0.768      -0.96831         0.832     -0.580359   search
      7        0.754745      -1.06331      0.845255     -0.513418   search
      9           0.736      -1.20704         0.864     -0.425868   search
     11         0.70949      -1.43013       0.89051     -0.315445   search
     13           0.672      -1.78834         0.928     -0.183784   search
     15        0.618981      -2.38951      0.981019    -0.0405454   search
     17           0.544      -3.45682         1.056     0.0915799   search
   
  Search for a zero in the interval [0.544, 1.056]:
   Func-count    x          f(x)             Procedure
     17           1.056     0.0915799        initial
     18         1.04279     0.0734457        interpolation
     19        0.990381    -0.0198942        interpolation
     20         1.00155    0.00308374        interpolation
     21         1.00005   0.000102524        interpolation
     22               1  -2.97656e-08        interpolation
     23               1   5.34328e-12        interpolation
     24               1             0        interpolation
   
  Zero found in the interval [0.544, 1.056]
\end{verbatim}
my$\_$fzero with 0.8:
\begin{verbatim}
  >> p2 = my_fzero(ftest,df_test, 0.8);
Search for an interval around 0.80 containing a sign change:
 Iter          a          f(a)          b           f(b)         Procedure
    1            0.8     -0.76032          0.8     -0.76032        initial
    2          0.799    -0.766382        0.801    -0.754286        search
    3          0.797    -0.778587        0.803    -0.742299        search
    4          0.793    -0.803331        0.807     -0.71865        search
    5          0.785    -0.854164        0.815    -0.672633        search
    6          0.769    -0.961359        0.831    -0.585585        search
    7          0.737     -1.19909        0.863    -0.430335        search
    8          0.673      -1.7781        0.927    -0.186948        search
    9          0.545     -3.44074        1.055    0.0902771        search
Search for a zero in the interval [0.545,1.055]
 Iter        x          f(x)             Procedure
    9           0.545      -3.44074        initial
   10        0.759271      -1.03026        newton
   11        0.901756     -0.273059        newton
   12        0.976456     -0.051088        newton
   13        0.998243   -0.00353567        newton
   14        0.999989  -2.14434e-05        newton
   15               1   -8.0459e-10        newton
   16               1             0        newton
Zero found in the interval [0.545000,1.055000]
\end{verbatim}
fzero with 1.6:
\begin{verbatim}
  >> p = fzero(ftest, 1.6,optimset('display','iter'));
 
Search for an interval around 1.6 containing a sign change:
 Func-count    a          f(a)             b          f(b)        Procedure
    1             1.6       0.05376           1.6       0.05376   initial interval
    3         1.55475     0.0707724       1.64525     0.0390246   search
    5           1.536     0.0783898         1.664     0.0336506   search
    7         1.50949     0.0896218       1.69051     0.0268048   search
    9           1.472      0.106162         1.728     0.0186348   search
   11         1.41898      0.129928       1.78102    0.00999714   search
   13           1.344      0.160816         1.856    0.00292407   search
   15         1.23796      0.185547       1.96204   5.46258e-05   search
   17           1.088      0.127631         2.112    -0.0013873   search
 
Search for a zero in the interval [1.088, 2.112]:
 Func-count    x          f(x)             Procedure
   17           2.112    -0.0013873        initial
   18         2.10099   -0.00101946        interpolation
   19         2.07071  -0.000351843        interpolation
   20         2.07071  -0.000351843        bisection
   21         2.06791  -0.000311721        interpolation
   22         2.04622  -9.85177e-05        interpolation
   23         2.04622  -9.85177e-05        bisection
   24         2.04264  -7.73958e-05        interpolation
   25         2.02969  -2.61616e-05        interpolation
   26         2.02969  -2.61616e-05        bisection
   27         2.02415   -1.4076e-05        interpolation
   28         2.01787  -5.70768e-06        interpolation
   29         2.01787  -5.70768e-06        bisection
   30         2.00974  -9.23062e-07        interpolation
   31         2.00822  -5.54788e-07        interpolation
   32         2.00598  -2.13457e-07        interpolation
   33         2.00598  -2.13457e-07        bisection
   34         2.00407  -6.76507e-08        interpolation
   35         2.00322  -3.35028e-08        interpolation
   36         2.00241  -1.39759e-08        interpolation
   37         2.00241  -1.39759e-08        bisection
   38         2.00164  -4.38337e-09        interpolation
   39          2.0013  -2.17827e-09        interpolation
   40         2.00097  -9.07505e-10        interpolation
   41         2.00097  -9.07505e-10        bisection
   42         2.00066  -2.83649e-10        interpolation
   43         2.00052  -1.41171e-10        interpolation
   44         2.00039  -5.87193e-11        interpolation
   45         2.00039  -5.87193e-11        bisection
   46         2.00026   -1.8332e-11        interpolation
   47         2.00021  -9.23706e-12        interpolation
   48         2.00015  -3.72324e-12        interpolation
   49         2.00015  -3.72324e-12        bisection
   50         2.00011  -1.19371e-12        interpolation
   51         2.00008  -5.96856e-13        interpolation
   52         2.00006  -2.27374e-13        interpolation
   53         2.00006  -2.27374e-13        bisection
   54         2.00004  -1.13687e-13        interpolation
   55         2.00003  -2.84217e-14        interpolation
   56         2.00002  -2.84217e-14        interpolation
   57         2.00002  -2.84217e-14        bisection
   58               2   2.84217e-14        interpolation
   59         2.00001  -2.84217e-14        bisection
   60               2   2.84217e-14        bisection
   61               2   2.84217e-14        interpolation
   62               2   5.68434e-14        bisection
   63         2.00001  -2.84217e-14        bisection
   64               2             0        interpolation
 
Zero found in the interval [1.088, 2.112]
\end{verbatim}
my$\_$fzero with 1.6:
\begin{verbatim}
  >> p2 = my_fzero(ftest,df_test, 1.6);
Search for an interval around 1.60 containing a sign change:
 Iter          a          f(a)          b           f(b)         Procedure
    1            1.6      0.05376          1.6      0.05376        initial
    2          1.599    0.0541126        1.601    0.0534086        search
    3          1.597     0.054821        1.603    0.0527091        search
    4          1.593    0.0562512        1.607    0.0513236        search
    5          1.585    0.0591639        1.615    0.0486079        search
    6          1.569    0.0651904        1.631    0.0434022        search
    7          1.537    0.0779761        1.663    0.0339262        search
    8          1.473     0.105714        1.727      0.01883        search
    9          1.345      0.16045        1.855   0.00298453        search
   10          1.089      0.12859        2.111  -0.00135078        search
Search for a zero in the interval [1.089,2.111]
 Iter        x          f(x)             Procedure
   10           1.089       0.12859        initial
   11             1.6       0.05376        bisection
   12         1.75273     0.0141947        newton
   13         1.83889     0.0040731        newton
   14         1.89357    0.00119204        newton
   15         1.92932    0.00035137        newton
   16         1.95296   0.000103876        newton
   17         1.96866   3.07475e-05        newton
   18         1.97911   9.10639e-06        newton
   19         1.98608   2.69767e-06        newton
   20         1.99072    7.9924e-07        newton
   21         1.99381   2.36803e-07        newton
   22         1.99588   7.01626e-08        newton
   23         1.99725   2.07887e-08        newton
   24         1.99817   6.15961e-09        newton
   25         1.99878    1.8251e-09        newton
   26         1.99919   5.40695e-10        newton
   27         1.99946   1.60213e-10        newton
   28         1.99964   4.74927e-11        newton
   29         1.99976   1.40403e-11        newton
   30         1.99984   4.26326e-12        newton
   31         1.99989   1.22213e-12        newton
   32         1.99993   3.69482e-13        newton
   33         1.99995   5.68434e-14        newton
   34         1.99996   8.52651e-14        newton
   35         1.99998             0        newton
Zero found in the interval [1.089000,2.111000]
\end{verbatim}

fzero with 2.4:
\begin{verbatim}
  >> p = fzero(ftest, 2.4,optimset('display','iter'));
 
Search for an interval around 2.4 containing a sign change:
 Func-count    a          f(a)             b          f(b)        Procedure
    1             2.4      -0.05376           2.4      -0.05376   initial interval
    3         2.33212    -0.0325926       2.46788    -0.0800034   search
    5           2.304    -0.0254981         2.496    -0.0920041   search
    7         2.26424    -0.0171609       2.53576     -0.109644   search
    9           2.208   -0.00860958         2.592     -0.134762   search
   11         2.12847   -0.00208539       2.67153     -0.166267   search
   13           2.016  -4.09495e-06         2.784     -0.185694   search
   14         1.85694    0.00286785         2.784     -0.185694   search
 
Search for a zero in the interval [1.85694, 2.784]:
 Func-count    x          f(x)             Procedure
   14         1.85694    0.00286785        initial
   15         1.87104    0.00210894        interpolation
   16         1.90978   0.000728292        interpolation
   17         1.90978   0.000728292        bisection
   18         1.91828   0.000542003        interpolation
   19         1.94266   0.000187926        interpolation
   20         1.94266   0.000187926        bisection
   21         1.95468   9.28755e-05        interpolation
   22         1.96608   3.89965e-05        interpolation
   23         1.96608   3.89965e-05        bisection
   24         1.98277   5.11528e-06        interpolation
   25         1.98521   3.23155e-06        interpolation
   26         1.98933   1.21567e-06        interpolation
   27         1.98933   1.21567e-06        bisection
   28         1.99256   4.11233e-07        interpolation
   29         1.99416   1.99233e-07        interpolation
   30         1.99562   8.38226e-08        interpolation
   31         1.99562   8.38226e-08        bisection
   32           1.997   2.69152e-08        interpolation
   33         1.99763   1.32716e-08        interpolation
   34         1.99823   5.54576e-09        interpolation
   35         1.99823   5.54576e-09        bisection
   36          1.9988   1.74651e-09        interpolation
   37         1.99905   8.66748e-10        interpolation
   38         1.99929   3.61325e-10        interpolation
   39         1.99929   3.61325e-10        bisection
   40         1.99952   1.13062e-10        interpolation
   41         1.99962   5.62181e-11        interpolation
   42         1.99971   2.34195e-11        interpolation
   43         1.99971   2.34195e-11        bisection
   44         1.99981   7.30438e-12        interpolation
   45         1.99985   3.60956e-12        interpolation
   46         1.99988   1.62004e-12        interpolation
   47         1.99988   1.62004e-12        bisection
   48         1.99992   4.54747e-13        interpolation
   49         1.99994   2.27374e-13        interpolation
   50         1.99995   8.52651e-14        interpolation
   51         1.99995   8.52651e-14        bisection
   52         1.99997   2.84217e-14        interpolation
   53         1.99997  -2.84217e-14        interpolation
   54         1.99997   2.84217e-14        bisection
   55         1.99997   2.84217e-14        bisection
   56         1.99997   2.84217e-14        bisection
   57         1.99997   2.84217e-14        bisection
   58         1.99997  -2.84217e-14        bisection
   59         1.99997   2.84217e-14        bisection
   60         1.99997   2.84217e-14        bisection
   61         1.99997             0        bisection
 
Zero found in the interval [1.85694, 2.784]
\end{verbatim}

my$\_$fzero with 2.4:
\begin{verbatim}
  >> p2 = my_fzero(ftest,df_test, 2.4);
  Search for an interval around 2.40 containing a sign change:
   Iter          a          f(a)          b           f(b)         Procedure
      1            2.4     -0.05376          2.4     -0.05376        initial
      2          2.399   -0.0534086        2.401   -0.0541126        search
      3          2.397   -0.0527091        2.403    -0.054821        search
      4          2.393   -0.0513236        2.407   -0.0562512        search
      5          2.385   -0.0486079        2.415   -0.0591639        search
      6          2.369   -0.0434022        2.431   -0.0651904        search
      7          2.337   -0.0339262        2.463   -0.0779761        search
      8          2.273     -0.01883        2.527    -0.105714        search
      9          2.145  -0.00298453        2.655     -0.16045        search
     10          1.889   0.00135078        2.911     -0.12859        search
  Search for a zero in the interval [1.889,2.911]
   Iter        x          f(x)             Procedure
     10           1.889    0.00135078        initial
     11         1.92631   0.000397975        newton
     12         1.95096    0.00011763        newton
     13         1.96734    3.4816e-05        newton
     14         1.97823   1.03109e-05        newton
     15         1.98549   3.05445e-06        newton
     16         1.99033   9.04937e-07        newton
     17         1.99355   2.68118e-07        newton
     18          1.9957   7.94409e-08        newton
     19         1.99713   2.35379e-08        newton
     20         1.99809   6.97418e-09        newton
     21         1.99873    2.0664e-09        newton
     22         1.99915   6.12232e-10        newton
     23         1.99943   1.81444e-10        newton
     24         1.99962    5.3717e-11        newton
     25         1.99975   1.59446e-11        newton
     26         1.99983   4.77485e-12        newton
     27         1.99989   1.36424e-12        newton
     28         1.99993   4.54747e-13        newton
     29         1.99995   1.13687e-13        newton
     30         1.99997   2.84217e-14        newton
     31         1.99998             0        newton
  Zero found in the interval [1.889000,2.911000]
\end{verbatim}

fzero with 3.2:
\begin{verbatim}
  >> p = fzero(ftest, 3.2,optimset('display','iter'));
 
Search for an interval around 3.2 containing a sign change:
 Func-count    a          f(a)             b          f(b)        Procedure
    1             3.2       0.76032           3.2       0.76032   initial interval
    3         3.10949      0.315445       3.29051       1.43013   search
    5           3.072      0.183784         3.328       1.78834   search
    7         3.01898     0.0405454       3.38102       2.38951   search
    8           2.944    -0.0915799       3.38102       2.38951   search
 
Search for a zero in the interval [2.944, 3.38102]:
 Func-count    x          f(x)             Procedure
    8           2.944    -0.0915799        initial
    9         2.96013    -0.0691692        interpolation
   10         3.00852     0.0175479        interpolation
   11         2.99873   -0.00253688        interpolation
   12         2.99996  -7.47481e-05        interpolation
   13               3   1.58347e-08        interpolation
   14               3  -2.21689e-12        interpolation
   15               3    1.7053e-13        interpolation
   16               3             0        interpolation
 
Zero found in the interval [2.944, 3.38102]
\end{verbatim}

my$\_$fzero with 3.2:
\begin{verbatim}
  >> p2 = my_fzero(ftest,df_test, 3.2);
  Search for an interval around 3.20 containing a sign change:
   Iter          a          f(a)          b           f(b)         Procedure
      1            3.2      0.76032          3.2      0.76032        initial
      2          3.199     0.754286        3.201     0.766382        search
      3          3.197     0.742299        3.203     0.778587        search
      4          3.193      0.71865        3.207     0.803331        search
      5          3.185     0.672633        3.215     0.854164        search
      6          3.169     0.585585        3.231     0.961359        search
      7          3.137     0.430335        3.263      1.19909        search
      8          3.073     0.186948        3.327       1.7781        search
      9          2.945   -0.0902771        3.455      3.44074        search
  Search for a zero in the interval [2.945,3.455]
   Iter        x          f(x)             Procedure
      9           2.945    -0.0902771        initial
     10           3.014     0.0293935        newton
     11         3.00065    0.00129456        newton
     12               3   2.91125e-06        newton
     13               3   1.50067e-11        newton
     14               3             0        newton
  Zero found in the interval [2.945000,3.455000]
\end{verbatim}
\\
One difference I see is the increment step size for finding interval.
My findinterval function will increase the interval search step in each new iteration, but this does not always happen in MATLAB fzero.
For example, when initial guess point is 0.8. The difference between iter 1 and iter 3 is 0.0226 and difference between iter 3 and iter 5 is 0.009 for MATLAB fzero.
My findinterval will always increase step size 2 times the previous iteration.
\\
Another difference I observe is that MATLAB fzero tend to choose bisection more frequently than my fzero. 
When our initial guess point is not so good (1.6, 2.4). MATLAB fzero needs more iteration to converge and many iteration ends up using bisection.
My fzero rarely uses bisection method. 
One explanation is that MATLAB fzero uses IQI if there are three distinct points and it will check the difference between these point.
If the distance is close then it uses bisection.
\\
The third difference I observe is that my fzero function converges faster than MATLAB fzero, even converges to a root with multiplicity of 3.
I think this is because Newton's Method has a larger coefficient for linear convergence than interpolation method.
It will still converge faster though they are both in linearly convergence.
\\
I think my implementation is better than MATLAB's one in this case. 
Because my function needs the first derivative of input function.
Newton's Method will converge faster than bisection and interpolation.
It may be worse if the input function is not differentiable.













\end{document}