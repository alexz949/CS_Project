\documentclass{article}
\title{CSC391 project2 report}
\author{Alex Zhang}
\date{Oct 2023}
\textwidth=16.00cm 
\textheight=22.00cm 
\topmargin=0.00cm
\oddsidemargin=0.00cm 
\evensidemargin=0.00cm 
\headheight=0cm 
\headsep=0.5cm
\textheight=610pt
\usepackage{graphicx}
\usepackage{multicol}

\graphicspath{ {./images/} }

\usepackage{latexsym,array,delarray,amsthm,amssymb,epsfig}
\usepackage{amsmath}
\usepackage{listings}
\lstset{
  basicstyle=\ttfamily,
  mathescape
}

\newcommand{\bmat}[1]{\begin{bmatrix} #1 \end{bmatrix}}
\newcommand{\mat}[1]{\mathbf{#1}}

\let\ds\displaystyle

\begin{document}
\maketitle
\section*{Part1}
\subsection*{Rotation}
For SIFT, I tried to rotate the image counterclockwise by 45 degree.
I think the result shows that SIFT is not completely rotation invariant. 
When running python rotation, the features for doing SIFT increase a lot if I rotate the image.
I cannot really tell 
`'
\subsection*{Low-light illumination}



I find out that ORB approach is more invariant when the brightness changed compared to SIFT algorithm
I will try with surf once I finish cmake
\end{document}